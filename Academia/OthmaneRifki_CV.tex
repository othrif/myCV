\documentclass[a4paper,10pt]{article}

%A Few Useful Packages
\usepackage{longtable}
\usepackage{tabularx}
\usepackage{marvosym}
\usepackage{fontspec} 					%for loading fonts
\usepackage{xunicode,xltxtra,url,parskip} 		%other packages for formatting
\RequirePackage{color,graphicx}
\usepackage[usenames,dvipsnames]{xcolor}
%\usepackage[big]{layaureo} 				%better formatting of the A4 page
% an alternative to Layaureo can be ** \usepackage{fullpage} **
\usepackage[cm]{fullpage}
\usepackage{supertabular} 				%for Grades
\usepackage{titlesec}					%custom \section
%%%%%%%%\usepackage{fontawesome}
% Line break in table cell
\newcommand{\specialcell}[2][c]{%
  \begin{tabular}[#1]{@{}l@{}}#2\end{tabular}}

\newcommand{\CC}{C\nolinebreak\hspace{-.05em}\raisebox{.4ex}{\tiny\bf +}\nolinebreak\hspace{-.10em}\raisebox{.4ex}{\tiny\bf +}}
\usepackage{tabularx,makecell}

%Setup hyperref package, and colours for links
\usepackage{hyperref}
\definecolor{linkcolour}{rgb}{0,0.2,0.6}
\hypersetup{colorlinks,breaklinks,urlcolor=linkcolour, linkcolor=linkcolour}

%FONTS
%\defaultfontfeatures{Mapping=tex-text} 						%%Commented out by Serhan
%\setmainfont[SmallCapsFont = Fontin SmallCaps]{Fontin} 		%%Commented out by Serhan

%CV Sections inspired by:
%http://stefano.italians.nl/archives/26
\titleformat{\section}{\Large\scshape\raggedright}{}{0em}{}[\titlerule]
\titlespacing{\section}{0pt}{3pt}{3pt} % Used to be : 0-3-3
%Tweak a bit the top margin
\addtolength{\voffset}{1cm}

%Italian hyphenation for the word: ''corporations''
%\hyphenation{im-pre-se}

%-------------WATERMARK TEST [**not part of a CV**]---------------
\usepackage[absolute]{textpos}

\setlength{\TPHorizModule}{30mm}
\setlength{\TPVertModule}{\TPHorizModule}
\textblockorigin{2mm}{0.65\paperheight}
\setlength{\parindent}{0pt}

%--------------------BEGIN DOCUMENT----------------------
\begin{document}

%%WATERMARK TEST [**not part of a CV**]---------------
%\font\wm=''Baskerville:color=787878'' at 8pt
%\font\wmweb=''Baskerville:color=FF1493'' at 8pt
%{\wm
%	\begin{textblock}{1}(0,0)
%		\rotatebox{-90}{\parbox{500mm}{
%			Typeset by Alaettin Serhan Mete with \XeTeX\  \today\ %for
%			%{\wmweb \href{http://www.aleplasmati.comuv.com}{aleplasmati.comuv.com}}
%		}
%	}
%	\end{textblock}
%}

\pagestyle{empty} % non-numbered pages

\font\fb=''[cmr10]'' %for use with \LaTeX command

%--------------------TITLE--------------------\vspace{3mm}
\par{\centering
		{\Huge \textsc{Othmane Rifki}
	}\bigskip\par}

%--------------------SECTIONS-----------------------------------
%Section: Personal Data
\section{Personal Data}

\begin{tabular}{rl}
    \textsc{Address:}   					& Bahrenfelder Chaussee 92, 22761 Hamburg, Germany		        \\
    \textsc{Phone:}     				    & +49 157 87 36 9651								            \\
    \textsc{E-mail:}     					& \href{mailto:othmane.rifki@cern.ch}{othmane.rifki@cern.ch}	\\
    \textsc{ORCID iD:}						& \href{https://orcid.org/0000-0002-9169-0793}{https://orcid.org/0000-0002-9169-0793}
\end{tabular}

%Section: Research Interests
%\vspace{3mm}
%\section{Research Interests}
%Exotic and SUSY searches for physics beyond the Standard Model with main focus on direct production of
%heavy gauge bosons, top squarks, charginos/neutralinos and sleptons.
%Currently working on the ATLAS experiment at the Large Hadron Collider and based at
%the European Organization for Nuclear Research, Geneva, Switzerland.

%Section: Positions Held
\vspace{3mm}
\section{Current Employment}
\begin{tabular}{r|p{15.5cm}}
\textsc{2017--\phantom{0000}}		&	Postdoctoral Research Fellow 	\\
							&		{\bf DESY}, Hamburg, Germany  						\\
%\multicolumn{2}{c}{} \\
%\textsc{2015--2017}				&	Research Assistant  			\\
%							&		{\bf University of Oklahoma}, Norman, OK, USA {\it and} 						\\
%							& 		{\bf CERN}, Switzerland 		\\
%\multicolumn{2}{c}{} \\
%\textsc{2014--2015}				&	Graduate Research Fellow  			\\
%							&		{\bf University of Oklahoma}, Norman, OK, USA {\it and} 						\\
%							&		{\bf Argonne National Laboratory}, Lemont, IL, USA 					\\
%\multicolumn{2}{c}{} \\
%\textsc{2012--2014}				&		Teaching Assistant, Department of Physics and Astronomy 			\\
%							&		{\bf University of Oklahoma}, Norman, OK, USA 						\\
\end{tabular}


%Section: Education
\vspace{3mm}
\section{Education}

\begin{tabular}{r | p{15.5cm}}
\textsc{2012--2017}			&  Ph.D. in \textsc{Physics} - \textbf{University of Oklahoma}, Norman, OK, USA \\
							& Supervisor: Brad Abbott \\ % \phantom{Ttle}
							& Thesis: {\it``Search for supersymmetry in final states with two same-sign or three leptons and jets with the ATLAS detector at the LHC''}, \href{https://cds.cern.ch/record/2289509/files/CERN-THESIS-2017-181.pdf}{CERN-THESIS-2017-181} \\
							%https://arxiv.org/pdf/1712.08677.pdf
\multicolumn{2}{c}{} \\
\textsc{2009--2012} 	    & B.Sc. in \textsc{Physics}, with Honors - \textbf{Drexel University}, Philadelphia, PA, USA	\\
							& Supervisors: Gordon Richards, Mitch Newcomer					\\
							& Thesis: \it{``\href{https://www.nhn.ou.edu/~othrif/LSST_Camera_Electronics.pdf}{Precision characterization of prototype electronics for readout of the 3.2 gigapixels Camera for the LSST}''} \\
\end{tabular}


%Section: Awards
\vspace{3mm}
\section{Awards}

\begin{tabular}{rp{15.5cm}}
\textsc{2019}       & {\em Bonus Merit for the contributions to the ATLAS detector upgrade} - DESY\\
\textsc{2019}       & {\em Dodge Award to invite PhD graduate back for a department colloquium} - University of Oklahoma\\
\textsc{2018}       & {\em Nielsen Prize for excellence in doctoral research} - University of Oklahoma\\
\textsc{2017}       & {\em Kalbfleisch Prize for a distinguished graduate student} - University of Oklahoma\\
\textsc{2015}       & {\em Graduate Student Senate Grant} - University of Oklahoma\\
\textsc{2014-2015}    & {\em ATLAS Analysis Support Center Graduate Fellowship} - Argonne National Laboratory\\
\textsc{2013}       & {\em Arts and Sciences Merit Scholarship} - University of Oklahoma\\
\textsc{2011}       & {\em Larson Physics Scholarship for outstanding student in experimental science} - Drexel University\\
\textsc{2009-2012}    & {\em Drexel University's Dean Scholarship} - Drexel University\\
\end{tabular}


%Section: Programming
\vspace{3mm}
\section{Computing Skills}
\begin{tabular}{rl}
Operating Systems:  & Mac OS X, Linux SLC5/SLC6/CC7/Ubuntu, Windows 10  \\
Programming:    & C, C++, Python, Bash scripting, \LaTeX        \\
Libraries:          & STL, Boost, Intel TBB \\
Software:           & Matlab, Mathematica, QT, ATLAS Athena, Pythia, MadGraph, Sherpa, ROOT   \\
Machine Learning:   & Pandas, Scikit-learn, Numpy, Scipy, Keras, TensorFlow, OpenCV     \\
% Grid computing (HTCondor, Rucio)
\end{tabular}


%Section: Responsability Roles
\vspace{3mm}
\section{Leadership Roles}
\begin{tabular}{r | p{12cm}}
\multicolumn{1}{r|}{Analysis Contact}			& ATLAS search for invisible VBF Higgs decays  \hfill (2018--\phantom{0000}) \\
\multicolumn{1}{r|}{Editorial Board Member}		& ATLAS search for electroweak SUSY production (3 publications)  \hfill (2017--2019)\\
\multicolumn{1}{r|}{JINST referee}		        & Invited referee in two publications \hfill (2018--2019)	\\
\multicolumn{1}{r|}{Coordinator}              & ATLAS SUSY fake objects review group  \hfill (2015--2018)         \\
\multicolumn{1}{r|}{On-call expert}              & ATLAS readout system and RoI builder operational support	\hfill (2015--2017)					\\
\multicolumn{1}{r|}{ATLAS shifter}			     & Shift leader, trigger, and run control shifts	\hfill (2015--2018)					\\
\end{tabular}


%Section: Research experience - Current
\vspace{3mm}

\section{Current Activities in ATLAS}
\begin{tabularx}{\textwidth}{>{\centering\arraybackslash}X p{.8\textwidth} }
{\bf Upgrade} &   \textbf{ATLAS upgrade for High-Luminosity LHC: Inner Strip tracking detector} \\
	& Built an automated robotic assembly system to pick and place modules using an advanced camera with magnifying optics running vision algorithms, custom-designed vacuum pickup tools, and a high precision glue dispensing system to achieve a module placement accuracy of less than 50 microns. Loading demonstration available here: \href{https://youtu.be/rU1wHZcM4Ng}{https://youtu.be/rU1wHZcM4Ng} \\
			& \begin{itemize}
			   \item Built the module loading station at DESY in close collaboration with TRIUMF
			   \item Corresponded with industry vendors to obtain the needed equipment with the desired specifications
			   \item Managed the purchases of all the necessary equipment for the project (250k Euros)
			   \item Oversaw the production of custom tools at the DESY workshop and by industrial manufacturers, and the assembly of the loading station in the DESY clean room
			   \item Harmonized the loading procedure and software with the other module loading sites (TRIUMF, IFIC, Freiburg)
         \item Coordinated the module reception tests (electrical and mechanical) and repairs of wirebonds with DESY technicians
         \item Successfully built and tested the first semi-electrical petal at DESY with which we passed the Final Design Review
			 \end{itemize} \\[-1.5ex]

{\bf Analysis} &  \textbf{Search for invisible Higgs boson decays in Vector Boson Fusion} \\
			   & Led the most sensitive channel to search for direct decays of the Higgs boson, and played the key role in improving the sensitivity
         by 30\% on the same data set as the last publication, and 50\% with the full Run-2 dataset despite the increase in pileup conditions impacting the forward objects identification. \\
			   & \begin{itemize}
         \item Coordinated the analysis activities and supervised the work of PhD students as Analysis Contact
			   \item Developed the full analysis software framework used by the analysis team and produced the Monte Carlo simulation for the signal and background processes
			   \item Established an Analysis Contact Expert (ACE) collaboration with three theorists to improve filtering algorithms in Sherpa to simulate more $V$ + jets processes with less CPU. Uncertainties from the limited amount of simulation was the leading uncertainty in the previous publication
			\end{itemize}
\end{tabularx}
\begin{tabularx}{\textwidth}{>{\centering\arraybackslash}X p{.8\textwidth} }
			& \begin{itemize}
				\item Developing a theoretical model to extrapolate between electroweak bosons to more accurately constrain the main analysis background in collaboration with the ACEs
				\item Optimized the event selection to increase the Higgs boson signal acceptance by 50\% on the same dataset compared to the previous publication
				\item Refined the lepton selection to reduce the background in the signal region by 31\% and increase acceptance of background processes in the control region by 34\%
				\item Improved the event categorization to improve sensitivity to the Higgs boson signal
			  \end{itemize}\\[-1.5ex]
\end{tabularx}


%Section: Research experience - Current
\vspace{3mm}
\section{Previous Activities in ATLAS}
\begin{tabularx}{\textwidth}{>{\centering\arraybackslash}X p{.8\textwidth} }
  {\sl 2015--2017} & {\bf Search for supersymmetry (SUSY) at 13 TeV with the ATLAS detector} \\
  & Performed a search for SUSY with multiple leptons and jets which benefits from a low
  standard model (SM) background allowing the physics analysis to reach the best sensitivity in scenarios with small
  mass differences between SUSY particles.\\[-1.5ex]
  & \begin{itemize}
  \item Identified uncovered regions of the SUSY parameter space and designed search regions to target them expanding the
    potential discovery reach of ATLAS
  \item Pioneered a search with a novel experimental signature of three leptons of the same electric charge to target a model
    proposed by ANL theorists Carlos Waigner and Ian Low
  \item Developed new data-driven analysis techniques to measure and validate the analysis backgrounds, in particular mis-identified
    leptons from experimental effects
  \item Developed and maintained the analysis software framework (\CC~and Python) used by the analysis team and generated the Monte Carlo
    simulation for the signal and background processes
  \end{itemize}\\[-1.5ex]
\end{tabularx}
\begin{tabularx}{\textwidth}{>{\centering\arraybackslash}X p{.8\textwidth} }
 {\sl 2016--2017} & {\bf Experimental backgrounds review group in SUSY} \\
  & Led a review team to evaluate the analysis techniques used to estimate backgrounds from mis-identified particles
  (leptons, photons, taus, and $b$-jets) due to instrumental effects in all ATLAS SUSY analyses.\\
  & \begin{itemize}
  \item Harmonized the procedures employed in estimating instrumental backgrounds across SUSY analyses
  \item Provided recommendations and software tools to measure these backgrounds
  \item Expanded the usage of a new data-driven method I developed for estimating mis-identified lepton background to other ATLAS analyses
  \end{itemize}\\
\end{tabularx}
\begin{tabularx}{\textwidth}{>{\centering\arraybackslash}X p{.8\textwidth} }
  {\sl 2014--2016} & {\bf Evolution of the Region of Interest Builder of the ATLAS detector}  \\
  & Migrated the functionality of the multi-card custom electronics
  Region of Interest Builder (RoIB), which seeds the retrieval of every event recorded by ATLAS to a single PCI-Express card
  hosted in a commodity computing node (PC RoIB). This evolution was undertaken to increase the system flexibility
  and reduce the operational overload associated with custom electronics. \\
  & \begin{itemize}
  \item Developed a multi-threaded \CC~ library that collects the ATLAS regions of interest (RoIs) identified by the
    first level trigger and forwards them at 100 kHz rate to a computing farm
  \item Used modern Intel resource manager tools to efficiently balance the sharing of memory usage between hardware resources
  \item Tested the RoIB performance system in a test setup at ANL and CERN during the R\&D phase of the project
  \item Installed the PC RoIB in the ATLAS detector and started taking data since early 2016
  \item Implemented diagnostic tools to monitor the RoIB system information in real time
  \end{itemize}\\
\end{tabularx}
\begin{tabularx}{\textwidth}{>{\centering\arraybackslash}X p{.8\textwidth} }
 {\sl 2014--2015} & {\bf First observation of the $t\bar{t}W$ and $t\bar{t}Z$ processes with the ATLAS detector} \\
%  Argonne National Laboratory
  & Performed the first measurement of the $t\bar{t}W$ and $t\bar{t}Z$ production cross-sections at 8 TeV, and played the key role in
  improving the sensitivity of the $t\bar{t}W$ cross section from 3$\sigma$ to 5$\sigma$ and in reducing the uncertainty on
  the measurement by 40\%.\\
  & \begin{itemize}
  \item Developed a new technique that exploits correlations between the event kinematics (missing transverse momentum and
    the multiplicity of jets and $b$-jets) to separate signal from background
  \item Quantified the SM backgrounds and calculated the experimental backgrounds of the analysis
  \item Evaluated the signal modeling uncertainties affecting the precision of the measurement
  \end{itemize}\\
\end{tabularx}
\begin{tabularx}{\textwidth}{>{\centering\arraybackslash}X p{.8\textwidth} }
  {\sl 2014--2015} & {\bf Photon production cross section measurement with the ATLAS detector}  \\
  & Observed photons above 1 TeV for the first time and provided the measurement data to test the precision of QCD modelling and reduce the
  Parton Distribution Function (PDF) uncertainties used in every analysis at the LHC. \\
  & \begin{itemize}
  \item Calculated the computing intensive higher order corrections to the photon production rate using Mira at ALCF with ANL
    physicist Sergei Chekanov allowing the generation of high statistics Monte Carlo simulation with a speed up factor of 1000
  \item Achieved an agreement between prediction and data over ten orders of magnitude
  \end{itemize}\\
\end{tabularx}

%Section: Research experience - Current
\vspace{3mm}
\section{Previous research experience}
\begin{tabularx}{\textwidth}{>{\centering\arraybackslash}X p{.8\textwidth} }

{\sl 2011--2012} & {\bf Precision characterization of the LSST camera readout electronics at UPenn}  \\

  & Participated in the assembly and testing of the electronics that read out the three billion pixels of the Large Synoptic
  Survey Telescope (LSST) camera. \\

  & \begin{itemize}
  \item Assembled hardware and tested software and firmware for readout tests of the silicon pixels of the LSST camera
  \item Developed analysis tools for precision characterization of the readout electronics in terms of gain, noise, and channel crosstalk
  \end{itemize} \\

  {\sl 2010--2011} &{\bf Assembly and characterization of the Double Chooz PMTs at Drexel University} \\
  & \begin{itemize}
  \item Set up a data acquisition system to read out Photomultiplier tubes (PMTs)
  \item Characterized the PMTs by calibrating them and measuring their gains, dark rates, and afterpulsing rates
  \item Built and tested voltage dividers for the PMTs
  \end{itemize}\\

\end{tabularx}

%Section: Teaching experience
\vspace{3mm}
\section{Teaching and Mentoring Experience}

\begin{tabular}{rp{15.5cm}}
\textsc{2018--\phantom{0000}} & {\bf Mentorship} of several Ph.D. students in the search for invisible VBF Higgs decays \\
\textsc{2019--\phantom{0000}} & {\bf Direct supervision} of two Ph.D. students (Pablo Rivadeneira, Alessia Renardi) \\
\textsc{2018--2019}           & {\bf Direct supervision} of Master's student (Janik von Ahnen) and summer student (Maren Stratmann)   \\
\textsc{2017--2018}           & {\bf Direct supervision} of Ph.D. student (Vincent Kitali)   \\
\textsc{2016}                 & {\bf Supervision} of two graduate students (Rishabh Jain and Joseph Lambert) in physics analysis  \\
\textsc{2015}                 & {\bf Support} of an undergraduate student (Taira Lamphere) in final year research project at OU \\
\textsc{2013}          & {\bf Course instructor} of electricity and magnetism for engineering students (1 semester)\\
\textsc{2012--2014}		      & {\bf Teaching Assistant} of general physics for engineers (4 semesters)\\
\end{tabular}

% Publications
\vspace{3mm}
\section{Journal Publications}
{ATLAS author since June 2015. Listed below are the publications to which I have significantly contributed}
\begin{itemize}
%	\item ATLAS Collaboration,
%	``Combination of searches for invisible Higgs boson decays with the ATLAS experiment'',
%	\href{https://journals.aps.org/prl/abstract/10.1103/PhysRevLett.122.231801}{Phys.\ Rev.\ Lett. {\bf 122}, 231801 (2019)}
	\item ATLAS Collaboration,
	``Search for invisible Higgs boson decays in vector boson fusion at $\sqrt{s}=13\ \mathrm{TeV}$ with the ATLAS detector'',
	\href{https://www.sciencedirect.com/science/article/pii/S0370269319302564}{Phys.\ Lett.\ {\bf B793} (2019) 499}
	\item ATLAS Collaboration,
	``Search for supersymmetry in final states with two same-sign or three leptons and jets using $36\ \mathrm{fb}^{-1}$ of $\sqrt{s}=13\ \mathrm{TeV}$ $pp$ collision data with the ATLAS detector'',
	\href{https://link.springer.com/article/10.1007/JHEP09(2017)084}{J.\ High\ Energy\ Phys.\ {\bf 09} (2017) 084}
	\item ATLAS Collaboration,
	``Search for supersymmetry at $\sqrt{s}=13\ \mathrm{TeV}$ in final states with jets and two same-sign leptons or three leptons with the ATLAS detector'',
	\href{https://link.springer.com/article/10.1140/epjc/s10052-016-4095-8}{Eur.\ Phys.\ J.\ {\bf C76} (2016) 259}
	\item O. Rifki et al.,
	``The evolution of the region of interest builder for the ATLAS experiment at CERN'',
	\href{https://iopscience.iop.org/article/10.1088/1748-0221/11/02/C02080}{JINST\ 11 C02080 (2016) }
	\item ATLAS Collaboration,
	``Measurement of the inclusive isolated prompt photon cross section in $pp$ collisions at $\sqrt{s}=8\ \mathrm{TeV}$ with the ATLAS detector'',
	\href{https://link.springer.com/article/10.1007/JHEP08(2016)005}{J.\ High\ Energy\ Phys.\ {\bf 06} (2016) 005 }
	\item ATLAS Collaboration,
	``Measurement of the $t\bar{t}W$ and $t\bar{t}Z$ production cross sections in $pp$ collisions at $\sqrt{s} = 8$ TeV with the ATLAS detector'',
	\href{https://link.springer.com/article/10.1007/JHEP08(2016)005}{J.\ High\ Energy\ Phys.\ {\bf 11} (2015) 172 }

\end{itemize}


% Conference Notes
\vspace{3mm}
\section{Conference Notes and Proceedings}

\begin{itemize}
	\item ATLAS Collaboration,
	``Search for invisible Higgs boson decays with vector boson fusion signatures with the ATLAS detector using an integrated luminosity of  $139\ \mathrm{fb}^{-1}$'',
	\href{https://atlas.web.cern.ch/Atlas/GROUPS/PHYSICS/CONFNOTES/ATLAS-CONF-2020-008/}{ATLAS-CONF-2020-008 (2020)}
	\item O. Rifki,
	``The ATLAS dataflow system in Run-2: Design and Performance'',
	38$^{\mathrm{th}}$ International Conference of High Energy Physics,
	\href{https://pos.sissa.it/282/240/pdf}{PoS ICHEP2016 (2017) 240}
	\item O. Rifki,
	``Search for supersymmetry at $\sqrt{s} = 13$ TeV in final states with two same-sign leptons or at least three leptons and jets using $pp$ collisions recorded with the ATLAS detector'',
	38$^{\mathrm{th}}$ International Conference of High Energy Physics,
	\href{https://pos.sissa.it/282/1126/pdf}{PoS ICHEP2016 (2017) 1126}
	\item ATLAS Collaboration,
	``Search for supersymmetry with two same-sign leptons or three leptons using $13.2\ \mathrm{fb}^{-1}$ of $\sqrt{s} = 13$ TeV $pp$ collision data collected by the ATLAS detector'',
	\href{https://atlas.web.cern.ch/Atlas/GROUPS/PHYSICS/CONFNOTES/ATLAS-CONF-2016-037/}{ATLAS-CONF-2016-037 (2016)}
	\item O. Rifki,
	``ATLAS searches for squarks and gluinos using leptons or multiple $b$-jets with $3.2\ \mathrm{fb}^{-1}$ of $pp$ collisions at $\sqrt{s} = 13$ TeV'',
	4$^{\mathrm{th}}$ Large Hadron Collider Physics Conference,
	\href{https://pos.sissa.it/276/236/pdf}{PoS LHCP2016 (2016) 236}
\end{itemize}

%Section: Talks and Posters
\vspace{3mm}
\section{Presentations} % Public
%\section{Talks}

\begin{tabular}{rp{15.5cm}}
\textsc{Apr. 2020}		& {\bf Quantum Universe Day} \hfill Hamburg, Germany\\
						& ``Probing dark matter with the Higgs boson via invisible decays''  	\\
						% https://indico.desy.de/indico/event/23731/session/0/contribution/1
\textsc{Apr. 2020}		& {\bf 89$^{\mathrm{th}}$ DESY Physics Research Committee Meeting} \hfill Hamburg, Germany\\
						& ``ATLAS group highlights at DESY''  	\\
						% https://indico.desy.de/indico/event/25538/
\textsc{Sep. 2019}		& {\bf OU Physics Department Colloquium} \hfill Norman, OK, USA\\
						& ``Higgs as a probe for dark matter''  	\\
						% https://www.nhn.ou.edu/news-and-events/events/colloquium224
\textsc{Sep. 2019}		& {\bf OU HEP Seminar} \hfill Norman, OK, USA\\
						& ``Invisible Higgs decay searches at the LHC''  	\\
						% LINK
\textsc{Sep. 2019}		& {\bf ANL HEP Division Seminar} \hfill Lemont, IL, USA \\
						& ``Searches for invisible Higgs decays with the ATLAS detector''  	\\
						% https://www.anl.gov/event/searches-for-invisible-higgs-decay-with-the-atlas-detector
\textsc{Aug. 2020}		& {\bf DESY HEP Experiment-Theory Seminar} \hfill Hamburg, Germany\\
						& ``Searches for invisible Higgs decays in the VBF channel with ATLAS ''  	\\
						%https://indico.desy.de/indico/event/23856/
\textsc{Jul. 2019}		& {\bf Division of Particles \& Fields of the American Physical Society} \hfill Boston, MA, USA \\
						& ``Searches for invisible Higgs decays with the ATLAS detector''  	\\
						% https://indico.cern.ch/event/782953/contributions/3455052/
\textsc{Jul. 2019}		& {\bf European Physical Society Conference on High Energy Physics} \hfill Ghent, Belgium \\
						& ``Dark matter searches with the ATLAS detector''  	\\
						% https://indico.cern.ch/event/577856/contributions/3420266/
						% https://cds.cern.ch/record/2683385
\textsc{Sep. 2018}		& {\bf DESY LHC Physics Seminar} \hfill Hamburg, Germany\\
						& ``Invisible Higgs decays in Vector Boson Fusion (VBF)''  	\\
						%https://indico.desy.de/indico/event/19964/
\textsc{Jun. 2018}		& {\bf 1$^{\mathrm{st}}$ African Conference on Fundamental Physics and Applications} \hfill Windhoek, Namibia \\
						& ``Dark Matter Searches at the LHC''  	\\
						% https://indico.cern.ch/event/667667/contributions/2951743/
						% https://cds.cern.ch/record/2624806
\textsc{May 2017}		& {\bf Phenomenology Symposium} \hfill Pittsburgh, PA, USA \\
						& ``Inclusive searches for squarks and gluinos with the ATLAS detector'' \\
						% https://indico.cern.ch/event/610112/contributions/2569884/
						% https://cds.cern.ch/record/2264385
\textsc{Aug. 2016}		& {\bf ANL HEP division seminar} \hfill Lemont, IL, USA \\
						& ``Search for supersymmetry at $\sqrt{s} = 13\ \mathrm{TeV}$ in leptonic final states with the ATLAS detector'' \\
						% REF?
\textsc{Aug. 2016}		& {\bf 38$^{\mathrm{th}}$ International Conference of High Energy Physics} \hfill Chicago, IL, USA \\
						& ``The ATLAS Data Flow system in Run-2: Design and Performance'' \\
						% ATL-DAQ-SLIDE-2016-508
\textsc{Aug. 2016}		& {\bf 38$^{\mathrm{th}}$ International Conference of High Energy Physics} \hfill Chicago, IL, USA \\
						& ``Search for supersymmetry at $\sqrt{s} = 13\ \mathrm{TeV}$ in final states with two same-sign leptons or at least three leptons and jets using $pp$ collisions recorded with the ATLAS detector''\\
						% ATL-PHYS-SLIDE-2016-495
\textsc{Jun. 2016}		& {\bf 4$^{\mathrm{th}}$ Large Hadron Collider Physics Conference} \hfill Lund, Sweden \\
						& ``ATLAS searches for squarks and gluinos using leptons or multiple b-jets with $3.2\ \mathrm{fb}^{-1}$ of $pp$ collisions at $\sqrt{s} = 13\ \mathrm{TeV}$''  (Poster)\\
						% ATL-PHYS-SLIDE-2016-360
\end{tabular}
\begin{tabular}{rp{15.5cm}}
\textsc{Jun. 2016}		& {\bf Low-x Workshop} \hfill Gyongyos, Hungary \\
						& ``Measurements of exclusive dilepton production at $7\ \mathrm{TeV}$ and $8\ \mathrm{TeV}$ with the ATLAS detector'' \\
						%  ATL-PHYS-SLIDE-2016-300
\textsc{Oct. 2015}		& {\bf Topical Workshop on Electronics for Particle Physics (TWEPP)} \hfill Lisbon, Portugal \\
						& ``The evolution of the Region of Interest Builder in the ATLAS experiment''  (Poster)\\
						% ATL-DAQ-SLIDE-2015-779
						% https://indico.cern.ch/event/357738/session/10/contribution/134
\textsc{Sep. 2015}		& {\bf European School of High Energy Physics} \hfill Bansko, Bulgaria\\
						& ``Measurement of the associated production cross section of a vector boson (W,Z) and top quark pair in $pp$ collisions at $\sqrt{s} = 8\ \mathrm{TeV}$ with the ATLAS detector'' (Poster)\\
						%https://espace.cern.ch/ESHEP2015/StudentPosters/Student%20Poster%20List/Forms/AllItems.aspx
\textsc{Mar. 2015}		& {\bf LHC Experiments Committee (LHCC)} at CERN \hfill Geneva, Switzerland  \\
						& ``The evolution of the Region of Interest Builder in the ATLAS experiment''  (Poster)\\
						%ATL-DAQ-SLIDE-2015-028
						%https://cds.cern.ch/record/1993248
\textsc{Oct. 2014}		& {\bf ANL Postdoctoral Research and Career Symposium} \hfill Lemont, IL, USA  \\
						& ``The evolution of the Region of Interest Builder in the ATLAS experiment''  (Poster) \\
\end{tabular}

%Section: ATLAS Talks
\vspace{3mm}
\section{Selected ATLAS Presentations}

\begin{tabular}{rp{15.5cm}}
\textsc{Feb. 2020}		& {\bf Final Design Review of ITk Strip Local Support Structures} \hfill Geneva, Switzerland\\
						& ``Module loading for the ITK end-cap petal cores''  	\\
						% https://indico.cern.ch/event/872236/#12-module-loading-petals
\textsc{Feb. 2020}		& {\bf ATLAS Collaboration Week} \hfill Geneva, Switzerland\\
						& ``Searches for dark matter''  	\\
						% https://indico.cern.ch/event/868980/contributions/3663511/
\textsc{Apr. 2019}		& {\bf ATLAS Upgrade Week} \hfill Geneva, Switzerland\\
						& ``Progress on end-cap module loading at DESY and IFIC''  	\\
						% https://indico.cern.ch/event/806682/#4-progress-on-ec-module-loadin
\textsc{Nov. 2019}		& {\bf ATLAS Upgrade Week} \hfill Geneva, Switzerland\\
						& ``Status of end-cap loading at DESY''  	\\
						% https://indico.cern.ch/event/858220/#1-status-of-endcap-loading-sit
\textsc{Oct. 2016}		& {\bf ATLAS Collaboration Week} \hfill Geneva, Switzerland\\
						& ``Performance of the PC based RoIB/HLTSV during 2016 Data-taking''  	\\
						% https://indico.cern.ch/event/577349/
\textsc{May 2016}		& {\bf ATLAS TDAQ week} \hfill Geneva, Switzerland\\
						& ``PC based RoIB: Installation, Operations, and Performance''  	\\
						 % https://indico.cern.ch/event/464852/
\end{tabular}
\begin{tabular}{rp{15.5cm}}
\textsc{Nov. 2015}		& {\bf High Level Trigger Supervisor and RoI Builder Software Review} \hfill Geneva, Switzerland\\
						& ``PC based RoIB performance''  	\\
						 % https://indico.cern.ch/event/461460/#2-roibuilder-software-incl-dis
\textsc{Jun. 2015}		& {\bf US ATLAS Workshop at UIUC} \hfill Champaign, IL, USA\\
						& ``Measurement of the inclusive isolated prompt photon cross section in $pp$ collisions at 8 TeV with the ATLAS detector using $20\ \mathrm{fb}^{-1}$''  	\\
						  %https://indico.cern.ch/event/388328/other-view?view=nicecompact#20150623.detailed
\textsc{Jun. 2015}		& {\bf US ATLAS Workshop at UIUC} \hfill Champaign, IL, USA\\
						& ``Simultaneous measurements of the $t\bar{t}W$ and $t\bar{t}Z$ production cross-sections using events with same-sign leptons from $pp$ collisions at 8 TeV using $20\ \mathrm{fb}^{-1}$''  	\\
						  %https://indico.cern.ch/event/388328/other-view?view=nicecompact#20150623.detailed
%\textsc{May 2015}		& {\bf Vector-Like Workshop at ANL} \hfill Lemont, IL, USA\\
%						& ``High $p_{\mathrm{T}}$ $b$-Tagging performance in Run-2 of the LHC''  	\\
%						  %https://indico.cern.ch/event/388328/other-view?view=nicecompact#20150623.detailed
\end{tabular}

%Section: Outreach
\vspace{3mm}
\section{Outreach Activities}

\begin{tabular}{rp{15.5cm}}
\textsc{2020}		& {\bf ATLAS website}: physics briefing on ``\href{http://atlas.cern/updates/physics-briefing/probing-dark-matter-higgs-boson}{probing dark matter with the Higgs boson}'' \\
\textsc{2020}		& {\bf High school student}: hosted and prepared projects for student to learn about particle physics\\
\textsc{2019}		& {\bf DESY behind the scenes}: organizer of ATLAS exhibition and poster on ``shedding light on the dark universe'' \\
\textsc{2017}		& {\bf DESY open day}: organizer of ATLAS exhibition and speaker ({\it +20,000 visitors})\\
\textsc{2016}		& {\bf Outreach during ICHEP}: Hands-on physics lectures in Chicago public libraries \\
\end{tabular}

%Section: Languages
\vspace{3mm}
\section{Languages}

\begin{tabular}{rl}
\textsc{English:}	& Fluent					\\
\textsc{French:}	& Native	\\
\textsc{Arabic:}	& Native    \\
\textsc{German:}	& Elementary Proficiency		\\
\end{tabular}

\end{document}



% Summer student - Maren Stratmann - https://www.desy.de/f/students/2019/reports/Maren.Stratmann.pdf
% prizes Homer Dodge prize OU
% outreach: DESY Open Days https://indico.desy.de/indico/event/23989/call-for-abstracts/18/
% outreach: https://www.facebook.com/ATLASexperiment/photos/a.239880036049433.51122.235921729778597/1094713383899423/?type=3&theater
%