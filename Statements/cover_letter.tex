\documentclass[a4paper]{article}

%\usepackage{lineno}

\usepackage[hyphenbreaks]{breakurl}
\usepackage[hyphens]{url}


\setlength{\topmargin}{-5mm}
\setlength{\textwidth}{7in}
\setlength{\oddsidemargin}{-8mm}
\setlength{\textheight}{9in}
\setlength{\footskip}{1in}

\newcommand{\ttbarV}{\ensuremath{t\bar{t}V}}
\newcommand{\ttbarW}{\ensuremath{t\bar{t}W}}
\newcommand{\ttbarZ}{\ensuremath{t\bar{t}Z}}

\newcommand{\ifb}{\ensuremath{\mathrm{fb}^{-1}}}

\begin{document}
%\linenumbers
\fontsize{12}{15}
%\fontsize{10}{13}
\selectfont
%\maketitle


%\noindent{\textbf{Lawrence Berkeley National Laboratory\\}}{Physics Division\\1 Cyclotron Rd, Berkeley, CA 94720\\ United States}
%\noindent{Othmane Rifki\\}
%\noindent{October 2016}

%\vspace{-5cm}
\noindent{Dear selection committee,}
\vspace{0.3cm}

Please accept this letter and CV as application for the postdoctoral position in New York University's High Energy Physics group.
For the past five years, I have been a PhD student with the experimental particle physics group at the University of Oklahoma 
while being based at Argonne National Laboratory (ANL) and CERN.
I conducted  my research in order to  better understand the fundamental interactions of elementary particles 
and to search for new physics phenomena by studying proton-proton collisions recorded by the ATLAS detector at the world's
most powerful particle accelerator, the Large Hadron Collider (LHC) at CERN.

During my PhD, I tested the validity of the Standard Model (SM) at energies never explored before, searched for new physics 
within the framework of supersymmetry (SUSY), and developed a system that processes every event recorded by ATLAS.
%I am fascinated by the predictive power of the Standard Model (SM) of particle physics that was successful in describing 
%elementary particle interactions over a wide energy scale. Yet, the SM is known to have limitations which points to 
%the existence of physics beyond the SM. I also enjoy building complex particle detectors that help us answer fundamental questions
%about the universe we live in. It is with this philosophy that I approached my PhD where I studied the theoretical foundations of the SM and 
%Supersymmetry (SUSY), 
As a result of this work, I contributed to eight publications and gave eleven public presentations (talks and posters) 
including a talk on the ATLAS dataflow system in Run 2 at ICHEP2016.
I made major contributions to the search for SUSY with two leptons of the same electric charge 
within ATLAS, and I gave three approval presentations of the analysis. 
%This analysis requires
%a robust understanding of the SM backgrounds involving processes with top quark, $W$ and $Z$ bosons, as well as backgrounds that arise
%from misidentified particles.
%Backgrounds that are essential for any new physics search and in which I am particularly expert.
%For instance, I have developed data-driven analysis techniques to estimate and validate these backgrounds and 
As a result of my expertise in data-driven background estimates, 
I led a review team to evaluate the analysis techniques used to estimate detector backgrounds across all the ATLAS SUSY analyses.
I have also expended the scope of models explored by ATLAS by introducing several new experimental 
signatures like a novel signature of three leptons of the same electric charge. 

In addition to analyzing physics data, I sought an opportunity to work on the trigger and data acquisition (TDAQ) system
of ATLAS at ANL. I received an ATLAS Support Center (ASC) fellowship to prototype and develop
the Region of Interest Builder (RoIB), the system that collects the regions of interest 
found by the hardware trigger of ATLAS and forwards them to the software High Level trigger (HLT).
I successfully installed and commissioned the system 
I built with the ANL colleagues which is currently used as baseline for data taking in ATLAS since the start of 2016.
In addition to the expert knowledge I acquired in TDAQ, I also got an exposure to ATLAS operations where I had 
to interact with both trigger and DAQ experts to understand the requirements of the different subsystems of TDAQ 
while integrating the RoIB.

The lack of evidence for new physics suggests that 
it might be hiding in experimentally hard-to-reach locations. For this reason, I would like to pursue
innovative ways to search for new physics which may entail development of new trigger strategies to better distinguish 
rare signals from the background. I believe this goal aligns well 
with NYU's plan to pursue searches for SUSY in non-standard channels and to work done the ATLAS trigger software. 
Furthermore, my experience in integrating the 
RoIB within the ATLAS TDAQ system has prepared me to be involved in trigger operations. 
NYU plays an important role in both searches for new physics and the ATLAS trigger, and I would be delighted to work,
learn, and grow as part of the group.

\vspace{0.25cm}

Sincerely,

\vspace{0.25cm}
Othmane Rifki



\end{document}
