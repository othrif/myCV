\documentclass[a4paper]{article}

%\usepackage{lineno}

\usepackage[hyphenbreaks]{breakurl}
\usepackage[hyphens]{url}
\usepackage{hyperref}
\usepackage{makecell}

\setlength{\topmargin}{0mm}
\setlength{\textwidth}{7in}
\setlength{\oddsidemargin}{-8mm}
\setlength{\textheight}{9in}
\setlength{\footskip}{1in}

\newcommand{\ttbarV}{\ensuremath{t\bar{t}V}}
\newcommand{\ttbarW}{\ensuremath{t\bar{t}W}}
\newcommand{\ttbarZ}{\ensuremath{t\bar{t}Z}}

\newcommand{\ifb}{\ensuremath{\mathrm{fb}^{-1}}}

\begin{document}
%\linenumbers
\fontsize{12}{15}
%\fontsize{10}{13}
\selectfont
%\maketitle


\null\hfill
\begin{tabular}{r}
  {\bf Othmane Rifki}  \\
  \href{mailto:othmane.rifki@cern.ch}{othmane.rifki@cern.ch} \\
  Bahrenfelder Chaussee 92\\
  22761 Hamburg, Germany \\
\makecell[r]{\\May 15, 2020}
\end{tabular}

\vspace{.2cm}
\noindent{\textbf{Postdoctoral Appointee at Argonne National Laboratory}} \\
%\vspace{.1cm}


\noindent{Dear Dr. Metcalfe and members of the selection committee,}
\vspace{0.3cm}

After obtaining my Ph.D. in August 2017 at the University of Oklahoma, I received a DESY Fellowship to work on both hardware and physics analysis.
I joined the large DESY involvement in the ITk strips community and initiated the project of building a module loading station in a newly commissioned clean room.
%This station will be used to load over one thousand silicon strip modules onto the local support structure of the
%ITk end-cap to be delivered by DESY.
 I successfully built this station, in close collaboration with engineers from DESY and researchers from other
institutes, and demonstrated that the station is ready for production at the module loading Final Design Review at CERN in February 2020.
This experience was invaluable in teaching me about silicon detectors and the testing of silicon modules, a key step before loading the modules.
In parallel to my detector work, I searched for dark matter production by directly measuring the invisible Higgs decays.
As analysis contact, I led this search to set the best LHC limit on invisible Higgs decays, extending the sensitivity of this fundamental probe for
dark matter by 50\%. With a conference note released last March 2020, I am now finalizing the paper with an important improvement to better control the main $Z\left(\nu\nu\right)$+jets background taking advantage of a collaboration with theorists that I established.

During my PhD, I made major contributions to the search for SUSY with two leptons of the same electric charge
 and expended the scope of models explored by introducing a novel signature of three leptons of the same electric charge.
As a result of my expertise in data-driven background estimates,
I led a review team to evaluate the analysis techniques used to estimate detector backgrounds across all the ATLAS SUSY analyses.
In addition to analyzing physics data, I evolved the Region of Interest Builder (RoIB), the system that collects the regions of interest
found by the hardware trigger of ATLAS and forwards them to the software High Level trigger (HLT), to a modern PC based system.
I successfully installed and commissioned the system I built with the ANL colleagues which has been used as baseline for data taking in ATLAS since the start of 2016.

Over the past years, I have accumulated a wide range of experience in data analysis, detector operations and development, taking up leadership roles as well as supervising students and engineers. The next years are critical to start production for pixel modules of the ITk
and explore innovative analysis strategies for Run-3 to search for new  physics that might be hiding in experimentally hard-to-reach locations.
I believe my background in the ITk silicon strip upgrade and my experience performing new physics searches have prepared me
well to succeed in the postdoctoral role at ANL.

I invite you to read my contributions and plans in the attached research statement and CV. Thank you for considering my application.

\vspace{0.25cm}

Sincerely,

\vspace{0.25cm}
Othmane Rifki



\end{document}
