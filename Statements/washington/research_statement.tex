\documentclass[a4paper]{article}
\topmargin-2.0cm

%\usepackage{lineno}
\usepackage{amsmath}
\usepackage{amssymb}
\usepackage{amsthm}
\usepackage{amscd}
\usepackage{amsfonts}
\usepackage{graphicx}%
\usepackage{fancyhdr}
\usepackage[hyphenbreaks]{breakurl}
\usepackage[hyphens]{url}

\newcommand{\ttbarV}{\ensuremath{t\bar{t}V}}
\newcommand{\ttbarW}{\ensuremath{t\bar{t}W}}
\newcommand{\ttbarZ}{\ensuremath{t\bar{t}Z}}
\newcommand{\ifb}{\ensuremath{\mathrm{fb}^{-1}}}

\setlength{\topmargin}{-15mm}
\setlength{\textwidth}{7in}
\setlength{\oddsidemargin}{-8mm}
\setlength{\textheight}{9in}
\setlength{\footskip}{1in}


%\title{\vspace{-3cm} Research Statement}
%\author{Othmane Rifki}%\\
%University of Oklahoma\\
%E-mail: \texttt{othmane.rifki@cern.ch}}
%\date{December 2016}


\begin{document}

\thispagestyle{fancy}\lhead{\large{Research Statement}} \rhead{\large{June 8, 2020}}
\chead{{\large{\bf Othmane Rifki}}} \lfoot{} \rfoot{\bf \thepage} \cfoot{}

%\linenumbers
\fontsize{11}{14}
\selectfont
%\maketitle

My research interests lie in exploring the Standard Model (SM) of particle physics and at the same time addressing questions not answered by the SM, such as the nature of dark matter (DM). Answering these questions requires innovative ideas to meet the ambitious experimental program of the LHC over the next decades by exploring uncharted territories, improving detector capabilities, and using state-of-the-art software technologies. After exploring several beyond the SM (BSM) physics searches with promptly decaying particles in ATLAS, I would now like to extend the discovery potential at the LHC by studying BSM long-lived signatures with the existing ATLAS detector. Given the data and computing challenges of the High-Luminosity LHC (HL-LHC), I am motivated to develop software tools particularly geared towards the Exabyte-scale problems we will face, using modern computing architectures and capitalizing on the latest developments in the data science and computer science communities in the context of the IRIS-HEP project. At the University of Washington, I would like to further my training and expertise, benefiting from and contributing to the expertise of the group in these areas.

\bigskip

So far, my research activities have spanned several areas in physics analysis and detector development. During my PhD, I performed the first searches for supersymmetry with early Run-2 data in leptonic final states after nearly doubling the center-of-mass energy\cite{SS3L}, and migrated the functionality of the Region of Interest Builder (RoIB), the interface that seeds the processing of every event in the high level trigger farm (HLT), from a VMEbus system to a PCIe card running in a commodity-computer. This new system has been the baseline for data taking in ATLAS since the start of 2016 \cite{proc-TWEPP2015-RoIB}. As a DESY fellow, I led the search for DM in invisible Higgs boson decays, setting the strongest existing limit to date, and built a module loading station for the new all-silicon Inner Tracker (ITk) of ATLAS for the HL-LHC.

\bigskip

At DESY, I have been deeply involved in the ITk strips upgrade \cite{ITKstrips}. I successfully developed a fully automated assembly system consisting of mounting and gluing modules with micrometer accuracy onto a petal core, the basic unit that forms the ITk end-cap disks, using a robotic gantry as a pick-and-place machine with advanced vision for identification of the sensor fiducial markers. I oversaw the purchasing of the equipment needed to build the station, totaling over $\$200$k, and exchanged with industry providers to test new solutions. I collaborated with other end-cap module loading sites to develop and standardize the module mounting procedure and adapt a common interface to streamline production. In the process, I gained substantial knowledge of detector development such as building and testing modules, operating industrial robotic assembly, and applying computer vision for pattern finding. I also improved my leadership and management skills by supervising a team of 5 including a PhD student, engineers, and technicians. In February 2020, I successfully presented the first fully loaded petal built at DESY using this automated procedure to the module loading Final Design Review, showing the petal assembly meets the desired specifications with high repeatability and increased capacity to make up for possible production delays. This experience combined with my expertise in detector readout and data acquisition systems will help me contribute to building a MATHUSLA demonstrator detector planned for the near term and a full-scale detector planned to be operational in the timescale of the HL-LHC.

\bigskip

In addition to detector development, I searched for DM production at the LHC with invisible Higgs boson decays. The Higgs boson total decay width is not well constrained with the visible decay modes, allowing up to 34\% of the decays to come from BSM processes. To that end, I performed a direct search of invisible Higgs decays in its most sensitive channel, vector boson fusion (VBF) production, where I identified key aspects to be improved leading the analysis to substantially increase sensitivity to this fundamental probe of DM \cite{vbfMET_CONF2020}. As analysis contact, I oversaw and guided the work of a team of graduate students and postdocs, including the day-to-day work of 5 graduate students, two of whom successfully defended their thesis. I also established a collaboration with 3 theorists to improve filtering algorithms used to generate Monte Carlo simulation of the main backgrounds and to enhance its theoretical accuracy and precision. In addition to developing a modular and scalable software package for the analysis, I recognized the importance of preserving the analysis workflow using new computing tools such as RECAST, a reproducible data analysis framework. I am using this framework to explore the Higgs sector by re-interpreting this search in terms of exotic decays of the Higgs boson to neutral long lived particles in the context of Higgs portal baryogenesis models. Over the coming months, if possible, I would like to continue leading the effort of this search to complete a legacy Run-2 paper and contribute to the combination effort of my analysis with other direct and indirect searches for invisible Higgs boson decays.

\bigskip

As there has been no definitive sign of new physics, it is important to fully exploit the LHC by looking in experimentally hard-to-reach locations that have not yet been probed in Run-2. I am incredibly excited to explore the search for long-lived particles that the University of Washington pioneered during Run-1 and Run-2 of the LHC. In addition to exploiting the utility of prompt associated objects to search for long-lived particles, I am also interested in improving standard reconstruction algorithms to ensure that tracking software and triggering are efficient at detecting displaced vertices. The higher computing cost and data size of large-radius tracking can be largely reduced by employing novel machine learning techniques to improve pattern recognition while keeping relaxed tracking impact parameter requirements. One example is to explore and extend the current outside-in tracking algorithms to use calorimeter cluster shapes more efficiently. Another is to tag long-lived particles by reconstructing displaced vertices looking at the hit patterns without doing any explicit track reconstruction. Including long-lived particle tracking into the standard reconstruction algorithms will have a considerable impact on searches for long-lived particles that decay within the inner tracking volume. Given my experience in software optimization for maximizing output within strict CPU limits in the HLT and my ambition to explore new machine learning techniques, I want to improve the long-lived particle tracking capability even further to expand the phase space probed by the LHC.

\bigskip

The coming few years are important to develop the software infrastructure needed to meet the computing challenges of the HL-LHC. I would like to explore areas where HEP can benefit from the development in the computer and data science communities to achieve highly performant analysis systems. This particularly entails the adoption of  industry standards and python based tools to efficiently process and analyze data. I also would like to continue exploring novel algorithmic approaches such as the use of machine learning to replace traditional algorithms for event selection and particle identification as well as learning more about differentiable programming and join the effort to use it for physics analysis. I would like to integrate these tools into our software and adapt our computing models to harness the full potential of these techniques. As part of the IRIS-HEP project, I would like to dedicate an important part of my time in learning and developing new analysis systems to further my professional development and training in these areas.

\bigskip

I plan to leverage my expertise in BSM searches, detector development, and data acquisition systems to search for long-lived particles in ATLAS and perform software R\&D for the IRIS-HEP project, while assuming more responsibilities and leadership roles as a senior postdoc. I would like to diversify my experience by getting involved in the prototyping of the new MATHUSLA surface detector for ultra long-lived particles. I intend to take part in the Snowmass effort to develop strategies for addressing the most important questions in particle physics. I enjoy working with and guiding students to advance their physics training and education, and connecting people with different areas of expertise in order to move a team and a project forward. The work I have done during my DESY fellowship as well as in my PhD have prepared me well to take on new challenges. The University of Washington plays a central role in searches for new physics, software R\&D, and detector developments, and I would be delighted to work on these areas, while I learn and grow as part of the group.




\Urlmuskip=0mu plus 1mu\relax
{%\small
\fontsize{11}{14}
\selectfont
\bibliography{research_statement}}{}
\bibliographystyle{unsrt}

\end{document}
