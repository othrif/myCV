\documentclass[a4paper]{article}

%\usepackage{lineno}

\usepackage[hyphenbreaks]{breakurl}
\usepackage[hyphens]{url}
\usepackage{hyperref}
\usepackage{makecell}

\setlength{\topmargin}{0mm}
\setlength{\textwidth}{7in}
\setlength{\oddsidemargin}{-8mm}
\setlength{\textheight}{9in}
\setlength{\footskip}{1in}

\newcommand{\ttbarV}{\ensuremath{t\bar{t}V}}
\newcommand{\ttbarW}{\ensuremath{t\bar{t}W}}
\newcommand{\ttbarZ}{\ensuremath{t\bar{t}Z}}

\newcommand{\ifb}{\ensuremath{\mathrm{fb}^{-1}}}

\begin{document}
%\linenumbers
\fontsize{12}{15}
%\fontsize{10}{13}
\selectfont
%\maketitle


\null\hfill
\begin{tabular}{r}
  {\bf Othmane Rifki}  \\
  \href{mailto:othmane.rifki@cern.ch}{othmane.rifki@cern.ch} \\
  Bahrenfelder Chaussee 92\\
  22761 Hamburg, Germany \\
\makecell[r]{\\May 15, 2020}
\end{tabular}

\vspace{.2cm}
\noindent{\textbf{Postdoctoral Appointee at Argonne National Laboratory}} \\
%\vspace{.1cm}


\noindent{Dear Dr. Metcalfe and members of the selection committee,}
\vspace{0.3cm}

Over the course of my PhD and DESY Fellowship, I conducted  my research in order to  better understand the fundamental interactions of elementary particles at energies never explored before and to search for new physics phenomena while developing new detector capabilities.


I joined the ITk strips community to build a module loading station at DESY in a newly commissioned clean room, in close collaboration with
engineers from DESY and researchers from other institutes.
%This station will be used to load over one thousand silicon strip modules onto the local support structure of the
%ITk end-cap to be delivered by DESY.
After completing the project, I demonstrated that the station is ready to load over one thousand silicon strip modules for the ITk
end-cap in the module loading Final Design Review at CERN in February 2020.
% I successfully built this station, in close collaboration with engineers from DESY and researchers from other
%institutes, and
This experience was invaluable in teaching me about silicon detectors and the testing of silicon modules, a key step before loading the modules.

In parallel to my detector work, I searched for dark matter production by directly measuring decays of the Higgs boson to invisible particles.
As analysis contact, I led this search to set the best LHC limit on invisible Higgs boson decays to date, extending the sensitivity of this fundamental probe for dark matter by 50\%. With a conference note released in April 2020, and building upon a collaboration with theorists that I established, I am currently finalizing the paper that will feature a better control of the main $Z\left(\nu\nu\right)$+jets background.

%During my PhD, I made major contributions to the search for SUSY with two leptons of the same electric charge
% and expended the scope of models explored by introducing a novel signature of three leptons of the same electric charge.
%As a result of my expertise in data-driven background estimates,
%I led a review team to evaluate the analysis techniques used to estimate detector backgrounds across all the ATLAS SUSY analyses.
%In addition to analyzing physics data, I evolved the Region of Interest Builder (RoIB), the system that collects the regions of interest
%found by the hardware trigger of ATLAS and forwards them to the software High Level trigger (HLT), to a modern PC based system.
%I successfully installed and commissioned the system I built with the ANL colleagues which has been used as baseline for data taking in ATLAS since %the start of 2016.


I would like to come to ANL to benefit and learn from the high energy physics group that has been instrumental to the success of the ATLAS experiment by leading major physics analyses and building key elements of the detector. I have a deep passion for and experience in both of these areas, and was
fortunate to spend one year at ANL as an ASC graduate fellow and work extensively with the group throughout my PhD.
This combination of factors makes me want to re-join the ANL group to work on exciting new physics analyses and building the next
generation silicon detector of the ATLAS experiment.


%Over the past years, I have accumulated a wide range of experience in data analysis, detector operations and development, taking up leadership roles as well as supervising students and engineers. The next years are critical to start production for pixel modules of the ITk
%and explore innovative analysis strategies for Run-3 to search for new  physics that might be hiding in experimentally hard-to-reach locations.
%I believe my background in the ITk silicon strip upgrade and my experience performing new physics searches have prepared me
%well to succeed in the postdoctoral role at ANL.

I invite you to read my contributions and plans in the attached research statement and CV. Thank you for your time and effort in considering my application.

\vspace{0.25cm}

Sincerely,

\vspace{0.25cm}
Othmane Rifki



\end{document}
