\documentclass[a4paper]{article}
\topmargin-2.0cm

%\usepackage{lineno}
\usepackage{amsmath}
\usepackage{amssymb}
\usepackage{amsthm}
\usepackage{amscd}
\usepackage{amsfonts}
\usepackage{graphicx}%
\usepackage{fancyhdr}
\usepackage[hyphenbreaks]{breakurl}
\usepackage[hyphens]{url}

\newcommand{\ttbarV}{\ensuremath{t\bar{t}V}}
\newcommand{\ttbarW}{\ensuremath{t\bar{t}W}}
\newcommand{\ttbarZ}{\ensuremath{t\bar{t}Z}}
\newcommand{\ifb}{\ensuremath{\mathrm{fb}^{-1}}}

\setlength{\topmargin}{-15mm}
\setlength{\textwidth}{7in}
\setlength{\oddsidemargin}{-8mm}
\setlength{\textheight}{9in}
\setlength{\footskip}{1in}


%\title{\vspace{-3cm} Research Statement}
%\author{Othmane Rifki}%\\
%University of Oklahoma\\
%E-mail: \texttt{othmane.rifki@cern.ch}}
%\date{December 2016}


\begin{document}

\thispagestyle{fancy}\lhead{\large{Research Statement}} \rhead{\large{May 2020}}
\chead{{\large{\bf Othmane Rifki}}} \lfoot{} \rfoot{\bf \thepage} \cfoot{}

%\linenumbers
\fontsize{11}{14}
\selectfont
%\maketitle

My research interests lie in exploring the Standard Model (SM) of particle physics through measurements of known and rare processes
to test the validity of the SM in new energy regimes and at the same time addressing questions not answered by the SM such as the nature of dark matter. Answering these questions are only possible through the improvement of detector capabilities to enhance the precision of our
measurements and to maximize the discovery potential of the ATLAS detector at the Large Hadron Collider (LHC) at CERN.

For the High-Luminosity LHC program planned to start in 2027, the ATLAS detector will be equipped with a new all-silicon Inner Tracker (ITk). DESY is deeply involved in the research and development of the silicon strip modules and will construct one full end-cap with $\sim 30 m^2$ of silicon.
The local support structures of the end-cap are disks built from petals with modules of different types and geometries.
I contributed significantly to the development of a fully automated petal assembly procedure which consists of mounting modules onto a petal core using a robotic gantry as a pick-and-place system with advanced vision for identification of the sensor fiducial markers.
Very recently, in February 2020, I successfully presented the first fully loaded petal built at DESY using this automated procedure with pattern finding to the end-cap local support Final Design Review, showing the petal assembly meets the desired specifications with high repeatability and increased capacity to make up for production delays that may arise in a complicated project.

In this work, I oversaw all activities related to building the module loading station from its inception in a newly commissioned clean room at DESY. I coordinated the ordering of the necessary equipment required to build the station (purchases totaled over $\$200$k) and exchanged with industry providers to test new solutions. I collaborated closely with other loading sites, in particular TRIUMF, to standardize the loading procedure and adapt a common interface to streamline production. Part of the assembly procedure is to verify that modules are qualified for loading through reception tests. I performed visual inspection of modules and facilitated wire-bond repairs prior to electrically testing the modules.
In the process, I gained substantial knowledge of detector development such as building and testing modules, operating industrial robotic assembly, and applying computer vision for pattern finding, and in parallel improved my leadership and management skills when supervising a team of 5 including a PhD student, engineers, and technicians.

My work in the strip ITk community have prepared me well to join the ANL group to work on the assembly and testing of the silicon pixel modules. I want to come to ANL to prepare the lab for preproduction and production of the ITk pixel modules expected during the upcoming years.
I would like to expand my detector experience to building modules, developing testing procedures, and characterizing pixel modules at test beams
irradiations needed for the success of the project.




I am particularly keen in machine learning algorithms, and software design for high-performance computers.


searches for signs of new physics, precision measurements of Higgs and the Standard Model, and working at the intersection of particle physics and machine learning to enhance the existing physics programs within ATLAS



the candidate is expected to participate in physics research carried out by the ANL ATLAS group. The group has strong involvement in Standard Model physics measurements, searches for new physics, Higgs measurements, and development of b-tagging algorithms. The Argonne group also has significant involvement in ATLAS detector operations, TDAQ upgrades, ATLAS core software development,



During my DESY Fellowship, I contributed significantly to the development of the ATLAS end-cap inner track strip detector (ITk) as part of the Phase-II upgrade. I now wish to assume responsibility in the pixel community to ...
I would like to join the Argonne group to deliver ... for the upcoming ITk milestones, while continuing to expand my expertise in detector development.
After leading the most sensitive search for dark matter in the direct decays of the Higgs boson to invisible particles, I want to join the
Argonne effort in ..., and make substantial improvements for both physics analysis as well as preparation for Run-3.


During my graduate education, I have been fortunate enough to take on a variety of analysis projects. I tested the SM by performing two measurements
with LHC Run-1 data. The first analysis was the cross-section measurement for inclusive prompt photon production,
which provides an important precision test for QCD modeling and helps reduce the Parton Distribution Function (PDF) uncertainties used in
every analysis at the LHC. I computed the differential cross sections for the production of inclusive photons
in the paper \cite{paper-2015-Photon} which were in agreement with data over ten orders of magnitude.
I continued exploring the SM by performing the cross-section measurement of the associated production of a top quark pair
and a vector boson (\ttbarV) \cite{paper-2015-ttV}.
I worked on the final state with two leptons of the same electric charge and $b$-jets which
drove the sensitivity of the \ttbarW~cross-section from an expected significance of 3 to 5 standard deviations improving upon the previous ATLAS result
and led to the first experimental observation of the \ttbarV~processes.
I played the key role in estimating the contribution from mis-identified leptons which represents the dominant background in this analysis.
After performing these SM measurements, I moved to searching for new physics.

One of the well-motivated theories for beyond the SM (BSM) physics is Supersymmetry (SUSY). SUSY predicts a superpartner
for every SM particle with the same quantum numbers, except spin. SUSY is appealing to me since it addresses the hierarchy problem related
to the large quantum corrections to the Higgs mass and can provide a candidate for a dark matter particle.

I searched for strongly
produced superpartners of quarks and gluons using events with two leptons of the same electric
charge or three leptons (SS3L) with the 13 TeV data collected by ATLAS
\cite{conf-2015-SS3L,paper-2015-SS3L,conf-2016-SS3L}.
This final state is characterized by a very low SM background
which allows for looser kinematic requirements than other BSM searches. As a result, the SS3L analysis has the unique potential to discover SUSY
in scenarios with small mass differences between SUSY particles (compressed scenarios) or to R-parity violating scenarios.
The main challenge in this search is to have a robust background modeling of very rare SM processes and
of mis-identified leptons due to detector effects. Given the expertise I gained from the \ttbarV~measurement and since the \ttbarV~processes constitute
the major irreducible background in signal regions with $b$-jets, I took on a leading role in developing data-driven analysis techniques
to estimate and validate the backgrounds. Furthermore, I identified previously uncovered areas of the SUSY parameter space that only our analysis
can explore and designed search regions to expand the scope of models explored by the ATLAS SUSY group.
For example, I designed a search with a novel experimental signature of three leptons of the same electric
charge targeting a well-motivated representation of a SUSY model featuring top quark superpartners proposed by Chicago theorist
Carlos Waigner et al. \cite{stop_3lss}.
As a result of my experience with background estimation, I am leading a team to evaluate the analysis techniques used to estimate
mis-identified objects (leptons, photons, etc.) across different analyses and provide recommendations and software tools to all
the analyses within the SUSY working group to estimate these backgrounds.

During the next years, I am interested to continue searching for new physics using the LHC data.
 With the current and relatively high exclusion limits on the masses of the colored superpartners,
we are approaching the moment where biggest discovery opportunities open up in the search for electroweak SUSY particles.
In particular, I am interested in the search for higgsino pair production since it is well motivated by naturalness arguments where
higgsinos are expected to be light and not far from the $W$, $Z$, and Higgs boson masses, and thus within reach at the LHC \cite{natural}.
I am also open to exploring other physics analyses either in the search for new physics using different experimental signatures or in SM measurements
as I have experience in both.

In order to detect the low cross-section signals I searched for, it is crucial to efficiently record the proton-proton collisions delivered by the LHC.
In the High Level Trigger (HLT), the trigger decisions are seeded by Regions of Interest (RoIs)
found by the custom hardware trigger system referred to as Level 1 (L1).
These RoIs are collected and assembled at 100 kHz rate by the Region of Interest Builder (RoIB)
\cite{proc-TWEPP2015-RoIB}.
%Since the RoIB seeds every event recorded by ATLAS, its efficient and successful operation is critical for data-taking.
%It was necessary to evolve the RoIB to maintain its performance in light of the increasingly difficult pileup and triggering conditions of Run-2.
I ensured a high data taking efficiency of the ATLAS detector by migrating the functionality of the RoIB from a custom multi-card crate of VME-based
electronics to a single custom PCI-Express (PCIe) card in a commodity-computer, in collaboration with Argonne National Laboratory (ANL).
The evolution was undertaken to maintain the performance of the RoIB in light of the increasingly difficult pileup and triggering conditions of Run-2.
I wrote a multi-threaded C++ software library that receives the custom inputs from L1 via optical links (S-Link),
 assembles these records at high rates into a single record, which is then forwarded to the HLT computing farm.
To improve upon the previous RoIB reporting capabilities,
I implemented diagnostic tools that publish system information to the online ATLAS monitoring services.
I made the initial tests of the new system at ANL in 2014 where I set up the testbed to
mimic the ATLAS infrastructure. Once validated, I prepared a similar setup in the ATLAS TDAQ testbed at CERN in 2015 which was running with conditions
closer to the online ATLAS system. After demonstrating the reliability and stability of the evolved RoIB, and with support
from the ATLAS TDAQ operations team,
I successfully installed and commissioned the evolved RoIB during the 2016 year end technical stop of the LHC.
The new system became the baseline system for ATLAS operations since the beginning of 2016.

I want to leverage my expertise in SUSY searches with leptons to continue searching for BSM physics or performing SM measurements. I also
aim at getting more responsibilities and take leadership roles within ATLAS. The work I have done on
leading the fake estimates group within SUSY can prepare me for such roles in the future.
Given my experiences with the development, commissioning, and supporting of the RoIB, I believe I can have a
significant contribution to the development and installation of hardware systems in ATLAS.
The University of Chicago plays a central role in both searches for new physics and hardware upgrades, and I would be delighted to work,
learn, and grow as part of the group.





\Urlmuskip=0mu plus 1mu\relax
{%\small
\fontsize{9}{4}
\selectfont
\bibliography{research_statement}}{}
\bibliographystyle{unsrt}

\end{document}
