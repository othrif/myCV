\documentclass[a4paper]{article}
\topmargin-2.0cm

%\usepackage{lineno}
\usepackage{amsmath}
\usepackage{amssymb}
\usepackage{amsthm}
\usepackage{amscd}
\usepackage{amsfonts}
\usepackage{graphicx}%
\usepackage{fancyhdr}
\usepackage[hyphenbreaks]{breakurl}
\usepackage[hyphens]{url}

\newcommand{\ttbarV}{\ensuremath{t\bar{t}V}}
\newcommand{\ttbarW}{\ensuremath{t\bar{t}W}}
\newcommand{\ttbarZ}{\ensuremath{t\bar{t}Z}}
\newcommand{\ifb}{\ensuremath{\mathrm{fb}^{-1}}}

\setlength{\topmargin}{-15mm}
\setlength{\textwidth}{7in}
\setlength{\oddsidemargin}{-8mm}
\setlength{\textheight}{9in}
\setlength{\footskip}{1in}


%\title{\vspace{-3cm} Research Statement}
%\author{Othmane Rifki}%\\
%University of Oklahoma\\
%E-mail: \texttt{othmane.rifki@cern.ch}}
%\date{December 2016}


\begin{document}

\thispagestyle{fancy}\lhead{\large{Research Statement}} \rhead{\large{May 15, 2020}}
\chead{{\large{\bf Othmane Rifki}}} \lfoot{} \rfoot{\bf \thepage} \cfoot{}

%\linenumbers
\fontsize{11}{14}
\selectfont
%\maketitle


My research interests lie in exploring the Standard Model (SM) of particle physics through measurements of known and rare processes
to test the validity of the SM in new energy regimes and at the same time address questions not answered by the SM such as the nature of
dark matter (DM), among others.
Answering these questions is made more possible by improving detector capabilities to enhance the precision of our measurements and to maximize the discovery potential of the ATLAS detector.

During my fellowship at DESY, I tackled these challenges by leading a search for DM in invisible Higgs boson decays, setting the strongest existing limit to date, and building a module loading station for the new all-silicon Inner Tracker (ITk) upgrade planned for the High-Luminosity LHC (HL-LHC). I would like to come to ANL to work on new physics analyses and the ITk, benefiting from and contributing to the expertise of the group in physics analysis and detector development.

The rate of invisible Higgs boson decays may be significantly enhanced beyond its SM value of $\sim0.1\%$ if the Higgs couples to DM particles.
To test this hypothesis, I coordinated the most sensitive search that drives the ATLAS direct probe of Higgs boson decays to invisible particles via the vector boson fusion (VBF) production mechanism \cite{vbfMET_CONF2020}. The VBF signature is a powerful probe for new physics as it exhibits a striking signature of two forward jets that have large invariant mass. This signature can be exploited to reject large SM backgrounds while retaining signal.
I established a collaboration between theorists and the ATLAS analysis team to improve
filtering algorithms used in the Monte Carlo generator, which resulted in the reduction of the largest uncertainty in the previous publication by 35\%. This work also enhanced the theoretical accuracy and precision of the simulation, which allowed us to better control the main $Z\left(\nu\nu\right)$+jets background.
%significantly enhanced simulated events of $W/Z$+jets production in the challenging analysis phase space of large invariant mass of the two-jet system.
I also oversaw the optimization of the event selection and analysis object definition to increase the Higgs boson signal acceptance by 50\% and reduce the background in the signal regions by 31\%. Through these improvements, my team and I significantly increased the sensitivity to this fundamental probe of dark matter by 54\%. In the upcoming months, I would like to continue leading the effort of this search to complete a
legacy Run-2 paper and contribute to the combination effort of my analysis with other direct and indirect searches for invisible Higgs boson decays.


Looking ahead, the next years are of great importance to prepare for Run-3 in terms of new analysis ideas, reconstruction techniques of physics objects, and trigger strategies. As there has been no definitive sign of new physics yet,
it is important to fully exploit the LHC by looking in experimentally hard-to-reach locations that haven't been probed.
One potential avenue that offers excellent sensitivity for well motivated scenarios is the search for long-lived particles
\cite{alimena2019searching}.
In analogy to the SM that have particles spanning a wide range of lifetimes, models beyond the SM (BSM) also predict new particles
with a variety of lifetimes, many of which can lead to macroscopic displacements from the primary interaction.
While the long-lived particles in the SM, such as b-hadrons, have well-understood experimental signatures, BSM long-lived particles
offer unusual signatures that may be rejected by standard reconstruction algorithms that are predominantly designed for prompt decays.
For instance, tracks corresponding to these hypothetical particles are reconstructed with a dedicated tracking configuration that only runs
on a filtered part of the main physics dataset due to significantly larger computing cost and data size. This cost can be significantly reduced
by employing novel machine learning techniques to make a better track seed selection, thus including long-lived particle tracking
into the standard reconstruction algorithms. This will have a considerable impact on many other downstream algorithms, such as tagging of different
physics objects (b-tagging, top tagging, etc.), and will expand the long-lived particles search program in ATLAS.
With my experience in optimizing software performance and ANL's expertise in machine learning algorithms and b-tagging,
I want to drive the long-lived particle tracking capability even further.

Another avenue through which to improve these searches is to exploit the utility of prompt associated objects.
For instance, I can explore the Higgs sector through the exotic decay of the Higgs boson to two long-lived neutral particles via the VBF production mode with large transverse momentum similar to the Higgs to invisible search I performed. In addition to using standard missing transverse momentum triggers, I am interested in investigating the use of more customized triggers that exploit the VBF topology of this signature.
For Run-4 and beyond, the larger silicon volume of the upgraded ATLAS ITk will extend the sensitivity to longer lifetimes of long-lived particles that decay within the tracker\cite{HLLHC_vertex}.
%and the low material budget will maintain low backgrounds for searches of these particles
As I want to work on the ITk detector upgrade, it will be complementary to my detector work to improve upon the upgrade tracking algorithms to cover a wider phase space for tracks with fewer silicon hits and much larger impact parameters consistent with displaced vertices that may originate from long-lived particles.
%The local support structures of the ITk end-cap are disks built from petals with modules of different types and geometries.

At DESY, I have been deeply involved in the upgrade of the silicon strip modules of the ITk \cite{ITKstrips}.
I led the DESY project to successfully develop a fully automated assembly system that places strip modules onto a petal, the basic unit that forms the ITk end-cap disks.
This system consists of mounting modules with less than 50 $\mu m$ lateral accuracy onto a petal core using a robotic gantry as a pick-and-place machine with advanced vision for identification of the sensor fiducial markers. I oversaw the purchasing of the equipment needed to build the station that totaled over $\$200$k, and exchanged with industry providers to test new solutions. I collaborated closely with other loading sites, in particular TRIUMF, to standardize the loading procedure and adapt a common interface to streamline production.
In the process, I gained substantial knowledge of detector development such as building and testing modules, operating industrial robotic assembly, and applying computer vision for pattern finding, and in parallel improved my leadership and management skills when supervising a team of 5 including a PhD student, engineers, and technicians. In February 2020, I successfully presented the first fully loaded petal built at DESY using this automated procedure to the module loading Final Design Review, showing the petal assembly meets the desired specifications with high repeatability and increased capacity to make up for possible production delays.

My work in the strip ITk community has prepared me well to join the ANL group to work on the assembly and testing of the silicon pixel modules.
I have already performed some of the tasks that will be required at ANL as part of the module loading procedure: for instance, prior to loading the modules on the local support structure, I verified that modules are qualified for assembly through visually inspecting the modules and
facilitating the wire-bonds repairs prior to electrically testing the modules.
With this experience, I want to learn from and contribute to ANL's work to prepare for production of the ITk pixel modules expected to ramp up during the upcoming years. I want to expand my detector experience to building and testing modules, as well as characterizing them in test beams.



I hope to leverage my expertise in BSM searches and detector work to continue searching for new physics and prepare for ITk
module production at ANL. I also aim to assume more responsibilities within ATLAS physics and upgrade communities. The work I have done
in my DESY fellowship as well as my PhD work have prepared me well to take on new challenges.
ANL plays a central role in both searches for new physics and hardware upgrades, and I would be delighted to work,
learn, and grow as part of the group.


% https://atlas.web.cern.ch/Atlas/GROUPS/PHYSICS/PAPERS/EXOT-2017-25


\Urlmuskip=0mu plus 1mu\relax
{%\small
\fontsize{11}{14}
\selectfont
\bibliography{research_statement}}{}
\bibliographystyle{unsrt}

\end{document}
