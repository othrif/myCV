\documentclass[a4paper]{article}
\topmargin-2.0cm

%\usepackage{lineno}
\usepackage{amsmath}
\usepackage{amssymb}
\usepackage{amsthm}
\usepackage{amscd}
\usepackage{amsfonts}
\usepackage{graphicx}%
\usepackage{fancyhdr}
\usepackage[hyphenbreaks]{breakurl}
\usepackage[hyphens]{url}

\newcommand{\ttbarV}{\ensuremath{t\bar{t}V}}
\newcommand{\ttbarW}{\ensuremath{t\bar{t}W}}
\newcommand{\ttbarZ}{\ensuremath{t\bar{t}Z}}
\newcommand{\ifb}{\ensuremath{\mathrm{fb}^{-1}}}

\setlength{\topmargin}{-15mm}
\setlength{\textwidth}{7in}
\setlength{\oddsidemargin}{-8mm}
\setlength{\textheight}{9in}
\setlength{\footskip}{1in}


%\title{\vspace{-3cm} Research Statement}
%\author{Othmane Rifki}%\\
%University of Oklahoma\\
%E-mail: \texttt{othmane.rifki@cern.ch}}
%\date{December 2016}


\begin{document}

\thispagestyle{fancy}\lhead{\large{Research Statement}} \rhead{\large{May 2020}}
\chead{{\large{\bf Othmane Rifki}}} \lfoot{} \rfoot{\bf \thepage} \cfoot{}

%\linenumbers
\fontsize{11}{14}
\selectfont
%\maketitle


My research interests lie in exploring the Standard Model (SM) of particle physics through measurements of known and rare processes
to test the validity of the SM in new energy regimes and at the same time addressing questions not answered by the SM.
These include questions such as the nature of dark matter (DM) and of the electroweak symmetry breaking (EWSB) mechanism, among others.
Answering these questions is made more possible by improving detector capabilities to enhance the precision of our measurements and to maximize the discovery potential of the ATLAS detector.
% at the Large Hadron Collider (LHC) at CERN.
During my fellowship at DESY, I tackled these challenges by leading a search for DM in invisible Higgs boson decays, setting the strongest existing limits to date, and building a module loading station for the new all-silicon Inner Tracker (ITk) upgrade planned for the High-Luminosity LHC (HL-LHC). I would like to come to ANL to continue working on these critical questions, benefiting from and contributing to the expertise of the group in physics analysis and detector development.



The rate of invisible Higgs boson decays may be significantly enhanced beyond its SM value of $\sim0.1\%$ if the Higgs couples to DM particles.
To test this hypothesis, I coordinated the most sensitive search that drives the ATLAS direct probe of Higgs boson decays to invisible particles via the vector boson fusion (VBF) production mechanism. I established a collaboration between theorists and the ATLAS analysis team to improve
filtering algorithms used in the Monte Carlo generator, which resulted in the reduction of the largest uncertainty in the previous publication by 35\%. This work also enhanced the theoretical accuracy and precision of the simulation, which allowed us to better control the main $Z\left(\nu\nu\right)$+jets background.
%significantly enhanced simulated events of $W/Z$+jets production in the challenging analysis phase space of large invariant mass of the two-jet system.
I also oversaw the optimization of the event selection and analysis object definition to increase the Higgs boson signal acceptance by 50\% and reduce the background in the signal regions by 31\%. Through these improvements, my team and I significantly increased the sensitivity to this fundamental probe of dark matter by 54\%.

The VBF production of the Higgs boson, characterized with jets in the forward regions of the detector, will benefit from the extended angular coverage of the ITk by enabling an improved jet reconstruction and better rejection of pile-up interactions.
DESY is deeply involved in the research and development of the silicon strip modules of the ITk and is responsible for constructing one full end-cap
with $\sim 30 m^2$ of silicon. The local support structures of the end-cap are disks built from petals with modules of different types and geometries.
I led the DESY project to successfully develop a fully automated petal assembly system in a newly commissioned clean room.
This system consists of mounting modules onto a petal core using a robotic gantry as a pick-and-place machine with advanced vision for identification of the sensor fiducial markers. I oversaw the purchasing of the equipment needed to build the station that totaled over $\$200$k, and exchanged with industry providers to test new solutions. I collaborated closely with other loading sites, in particular TRIUMF, to standardize the loading procedure and adapt a common interface to streamline production.
Very recently, in February 2020, I successfully presented the first fully loaded petal built at DESY using this automated procedure to the end-cap local support Final Design Review, showing the petal assembly meets the desired specifications with high repeatability and increased capacity to make up for production delays that may arise.


In the process, I gained substantial knowledge of detector development such as building and testing modules, operating industrial robotic assembly, and applying computer vision for pattern finding, and in parallel improved my leadership and management skills when supervising a team of 5 including a PhD student, engineers, and technicians.

My work in the strip ITk community have prepared me well to join the ANL group to work on the assembly and testing of the silicon pixel modules.
The module loading procedure at DESY required me to verify that modules are qualified for assembly through reception tests. I performed visual inspection of modules and facilitated wire-bond repairs prior to electrically testing the modules.
With this experience, I want to learn from and contribute to ANL's work to prepare for production of the ITk pixel modules expected to ramp up during the upcoming years. I want to expand my detector experience to building modules, developing testing procedures, and characterizing pixel modules at test beams irradiations needed for the success of the project.



The increased sensitivity
to VBF topologies will also

a high signal-to-background ratio can be achieve
improving this result will require developing new detector capabilities such as the extended rapidity

I am particularly keen in machine learning algorithms, and software design for high-performance computers.


searches for signs of new physics, precision measurements of Higgs and the Standard Model, and working at the intersection of particle physics and machine learning to enhance the existing physics programs within ATLAS


the candidate is expected to participate in physics research carried out by the ANL ATLAS group. The group has strong involvement in Standard Model physics measurements, searches for new physics, Higgs measurements, and development of b-tagging algorithms. The Argonne group also has significant involvement in ATLAS detector operations, TDAQ upgrades, ATLAS core software development,


Conclusion:
During my DESY Fellowship, I contributed significantly to the development of the ATLAS end-cap inner track strip detector (ITk) as part of the Phase-II upgrade. I now wish to assume responsibility in the pixel community to ...
I would like to join the Argonne group to deliver ... for the upcoming ITk milestones, while continuing to expand my expertise in detector development.
After leading the most sensitive search for dark matter in the direct decays of the Higgs boson to invisible particles, I want to join the
Argonne effort in ..., and make substantial improvements for both physics analysis as well as preparation for Run-3.


I want to leverage my expertise in SUSY searches with leptons to continue searching for BSM physics or performing SM measurements. I also
aim at getting more responsibilities and take leadership roles within ATLAS. The work I have done on
leading the fake estimates group within SUSY can prepare me for such roles in the future.
Given my experiences with the development, commissioning, and supporting of the RoIB, I believe I can have a
significant contribution to the development and installation of hardware systems in ATLAS.
The University of Chicago plays a central role in both searches for new physics and hardware upgrades, and I would be delighted to work,
learn, and grow as part of the group.





\Urlmuskip=0mu plus 1mu\relax
{%\small
\fontsize{9}{4}
\selectfont
\bibliography{research_statement}}{}
\bibliographystyle{unsrt}

\end{document}
