\documentclass[a4paper]{article}
\topmargin-2.0cm

%\usepackage{lineno}
\usepackage{amsmath}
\usepackage{amssymb}
\usepackage{amsthm}
\usepackage{amscd}
\usepackage{amsfonts}
\usepackage{graphicx}%
\usepackage{fancyhdr}
\usepackage[hyphenbreaks]{breakurl}
\usepackage[hyphens]{url}

\newcommand{\ttbarV}{\ensuremath{t\bar{t}V}}
\newcommand{\ttbarW}{\ensuremath{t\bar{t}W}}
\newcommand{\ttbarZ}{\ensuremath{t\bar{t}Z}}
\newcommand{\ifb}{\ensuremath{\mathrm{fb}^{-1}}}

\setlength{\topmargin}{-15mm}
\setlength{\textwidth}{7in}
\setlength{\oddsidemargin}{-8mm}
\setlength{\textheight}{9in}
\setlength{\footskip}{1in}


%\title{\vspace{-3cm} Research Statement}
%\author{Othmane Rifki}%\\
%University of Oklahoma\\
%E-mail: \texttt{othmane.rifki@cern.ch}}
%\date{December 2016}


\begin{document}

\thispagestyle{fancy}\lhead{\large{Research Statement}} \rhead{\large{May 2020}}
\chead{{\large{\bf Othmane Rifki}}} \lfoot{} \rfoot{\bf \thepage} \cfoot{}

%\linenumbers
\fontsize{11}{14}
\selectfont
%\maketitle


My research interests lie in exploring the Standard Model (SM) of particle physics through measurements of known and rare processes
to test the validity of the SM in new energy regimes and at the same time addressing questions not answered by the SM.
These include questions such as the nature of dark matter (DM) and of the electroweak symmetry breaking (EWSB) mechanism, among others.
Answering these questions is made more possible by improving detector capabilities to enhance the precision of our measurements and to maximize the discovery potential of the ATLAS detector.

During my fellowship at DESY, I tackled these challenges by leading a search for DM in invisible Higgs boson decays, setting the strongest existing limit to date, and building a module loading station for the new all-silicon Inner Tracker (ITk) upgrade planned for the High-Luminosity LHC (HL-LHC). I would like to come to ANL to work on new physics analyses and ITk, benefiting from and contributing to the expertise of the group in physics analysis and detector development.

The rate of invisible Higgs boson decays may be significantly enhanced beyond its SM value of $\sim0.1\%$ if the Higgs couples to DM particles.
To test this hypothesis, I coordinated the most sensitive search that drives the ATLAS direct probe of Higgs boson decays to invisible particles via the vector boson fusion (VBF) production mechanism. The VBF signature is a powerful probe for new physics as it exhibits a striking signature of two forward jets that have large invariant mass. This signature can be exploited to reject large SM backgrounds while retaining signal.
I established a collaboration between theorists and the ATLAS analysis team to improve
filtering algorithms used in the Monte Carlo generator, which resulted in the reduction of the largest uncertainty in the previous publication by 35\%. This work also enhanced the theoretical accuracy and precision of the simulation, which allowed us to better control the main $Z\left(\nu\nu\right)$+jets background.
%significantly enhanced simulated events of $W/Z$+jets production in the challenging analysis phase space of large invariant mass of the two-jet system.
I also oversaw the optimization of the event selection and analysis object definition to increase the Higgs boson signal acceptance by 50\% and reduce the background in the signal regions by 31\%. Through these improvements, my team and I significantly increased the sensitivity to this fundamental probe of dark matter by 54\%. In the upcoming months, I would like to continue leading the effort of the search and combination of invisible Higgs decays until the completion of a legacy full Run-2 paper.

%Prior to this analysis,
%I searched for strongly produced superpartners of quarks and gluons using events with two leptons of the same electric
%charge or three leptons (SS3L) with the 13 TeV data collected by ATLAS \cite{conf-2015-SS3L,paper-2015-SS3L,conf-2016-SS3L}.
%This final state is characterized by a very low SM background
%which allows for looser kinematic requirements than other BSM searches. As a result, the SS3L analysis has the unique potential to discover SUSY
%in scenarios with small mass differences between SUSY particles (compressed scenarios) or to R-parity violating scenarios.
%The main challenge in this search is to have a robust background modeling of very rare SM processes and
%of mis-identified leptons due to detector effects. Given the expertise I gained from the \ttbarV~measurement and since the \ttbarV~processes %constitute
%the major irreducible background in signal regions with $b$-jets, I took on a leading role in developing data-driven analysis techniques
%to estimate and validate the backgrounds. Furthermore, I identified previously uncovered areas of the SUSY parameter space that only our analysis
%can explore and designed search regions to expand the scope of models explored by the ATLAS SUSY group.
%For example, I designed a search with a novel experimental signature of three leptons of the same electric
%charge targeting a well-motivated representation of a SUSY model featuring top quark superpartners


%Looking ahead, I would like to capitalize on my expertise working with forward jets in the VBF final state and my previous analysis
%searching for supersymmetry in same-sign leptons to contribute to the measurement of the same-sign $W^{\pm}W^{\pm}$ production, a key process in understanding the nature of electroweak symmetry breaking. [To complete etc...]


% and the VBS in same sign WWW, are both
The VBF production of the Higgs boson  characterized with jets in the forward regions of the detector, will benefit from the extended angular coverage of the ITk by enabling an improved jet reconstruction and better rejection of pile-up interactions.
DESY is deeply involved in the research and development of the silicon strip modules of the ITk and is responsible for constructing one full end-cap
with $\sim 30$ m$^2$ of silicon. The local support structures of the end-cap are disks built from petals with modules of different types and geometries.
I led the DESY project to successfully develop a fully automated petal assembly system in a newly commissioned clean room.
This system consists of mounting modules onto a petal core using a robotic gantry as a pick-and-place machine with advanced vision for identification of the sensor fiducial markers. I oversaw the purchasing of the equipment needed to build the station that totaled over $\$200$k, and exchanged with industry providers to test new solutions. I collaborated closely with other loading sites, in particular TRIUMF, to standardize the loading procedure and adapt a common interface to streamline production.
In the process, I gained substantial knowledge of detector development such as building and testing modules, operating industrial robotic assembly, and applying computer vision for pattern finding, and in parallel improved my leadership and management skills when supervising a team of 5 including a PhD student, engineers, and technicians. In February 2020, I successfully presented the first fully loaded petal built at DESY using this automated procedure to the end-cap local support Final Design Review, showing the petal assembly meets the desired specifications with high repeatability and increased capacity to make up for production delays that may arise.

My work in the strip ITk community has prepared me well to join the ANL group to work on the assembly and testing of the silicon pixel modules.
I have already performed some of the tasks that will be required at ANL as part of the module loading procedure: for instance, prior to loading the modules on the local support structure, I verified that modules are qualified for assembly through visually inspecting the modules and
facilitating the wire-bonds repairs prior to electrically testing the modules.
With this experience, I want to learn from and contribute to ANL's work to prepare for production of the ITk pixel modules expected to ramp up during the upcoming years. I want to expand my detector experience to building and testing modules, as well as characterizing them in test beams.

%The next years are of great importance to finalizing analyis the full run-2 dataset and prepare for run-3 analyses. In light of no new phyiscs, it is important to explore more difficult final states and come up with innovative ideas. This is an area where machine learning will help.

I hope to leverage my expertise in new physics searches and detector work to continue searching for BSM physics and prepare for ITk
module production at ANL. I also aim to assume more responsibilities within ATLAS physics and upgrade communities. The work I have done
in my DESY fellowship as well as my PhD work have prepared me well to take on new challenges.
ANL plays a central role in both searches for new physics and hardware upgrades, and I would be delighted to work,
learn, and grow as part of the group.



%\Urlmuskip=0mu plus 1mu\relax
%{%\small
%\fontsize{9}{4}
%\selectfont
%\bibliography{research_statement}}{}
%\bibliographystyle{unsrt}

\end{document}
