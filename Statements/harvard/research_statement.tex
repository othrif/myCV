\documentclass[a4paper]{article}
\topmargin-2.0cm

%\usepackage{lineno}
\usepackage{amsmath}
\usepackage{amssymb}
\usepackage{amsthm}
\usepackage{amscd}
\usepackage{amsfonts}
\usepackage{graphicx}%
\usepackage{fancyhdr}
\usepackage[hyphenbreaks]{breakurl}
\usepackage[hyphens]{url}

\newcommand{\ttbarV}{\ensuremath{t\bar{t}V}}
\newcommand{\ttbarW}{\ensuremath{t\bar{t}W}}
\newcommand{\ttbarZ}{\ensuremath{t\bar{t}Z}}
\newcommand{\ifb}{\ensuremath{\mathrm{fb}^{-1}}}

\setlength{\topmargin}{-15mm}
\setlength{\textwidth}{7in}
\setlength{\oddsidemargin}{-8mm}
\setlength{\textheight}{9in}
\setlength{\footskip}{1in}


%\title{\vspace{-3cm} Research Statement}
%\author{Othmane Rifki}%\\
%University of Oklahoma\\
%E-mail: \texttt{othmane.rifki@cern.ch}}
%\date{December 2016}


\begin{document}

\thispagestyle{fancy}\lhead{\large{Research Statement}} \rhead{\large{June 8, 2020}}
\chead{{\large{\bf Othmane Rifki}}} \lfoot{} \rfoot{\bf \thepage} \cfoot{}

%\linenumbers
\fontsize{11}{14}
\selectfont
%\maketitle

My research interests lie in exploring the Standard Model (SM) of particle physics and at the same time addressing questions not answered by the SM, such as the nature of dark matter (DM). Answering these questions requires innovative ideas to meet the ambitious experimental program of the LHC over the next decades by exploring uncharted territories, improving detector capabilities, and using state-of-the-art software technologies. After exploring several beyond the SM (BSM) physics searches with promptly decaying particles in ATLAS, I would now like to extend the discovery potential at the LHC by studying BSM long-lived signatures. Given my detector experience in the new ATLAS inner tracker (ITk), I would like to play a leading role in constructing the ITk barrel strips detector. I would like to benefit from and contribute to the expertise of the Harvard group in these areas.



\bigskip

So far, my research activities have spanned several areas in physics analysis and detector development. During my PhD, I performed the first searches for supersymmetry with early Run-2 data in leptonic final states after nearly doubling the center-of-mass energy\cite{SS3L}, and migrated the functionality of the Region of Interest Builder (RoIB), the interface that seeds the processing of every event in the high level trigger farm (HLT), from a VMEbus system to a PCIe card running in a commodity-computer\cite{proc-TWEPP2015-RoIB}. This new system has been the baseline for data taking in ATLAS since the start of 2016. As a DESY fellow, I led the search for DM in invisible Higgs boson decays, setting the strongest existing limit to date, and built a module loading station for the ITk end-cap strips detector.

\bigskip

At DESY, I have been deeply involved in the ITk strips upgrade \cite{ITKstrips}. I successfully developed a fully automated assembly system consisting of mounting and gluing modules with micrometer accuracy onto a petal core, the basic unit that forms the ITk end-cap disks, using a robotic gantry as a pick-and-place machine with advanced vision for identification of the sensor fiducial markers. I oversaw the purchasing of the equipment needed to build the station, totaling over $\$200$k, and exchanged with industry providers to test new solutions. I collaborated with other end-cap module loading sites to develop and standardize the module mounting procedure and adapt a common interface to streamline production. In the process, I gained substantial knowledge of detector development such as building and testing modules, operating industrial robotic assembly, and applying computer vision for pattern finding. I also improved my leadership and management skills by supervising a team of 5 including a PhD student, engineers, and technicians. In February 2020, I successfully presented the first fully loaded petal built at DESY using this automated procedure to the module loading Final Design Review, showing the petal assembly meets the desired specifications with high repeatability and increased capacity to make up for possible production delays.

\bigskip

My work in the strip ITk community has prepared me well to join the Harvard group to work on the construction of the ITk barrel strips detector.
There is a large overlap between the work I have done already in the end-cap and the projects being pursed in the barrel. Since the next year is critical to ramp-up production of barrel staves, I am well positioned to rapidly contribute to the project. I have already performed some of the tasks that will be required such as performing electrical and thermo-mechanical tests of modules and assembled staves. The experience I gained from working in the ITk end-cap combined with my expertise in detector readout and data acquisition systems will help me take a leading role in building the ITk barrel strips detector.

\bigskip

In addition to detector development, I searched for DM production at the LHC with invisible Higgs boson decays. The Higgs boson total decay width is not well constrained with the visible decay modes, allowing up to 34\% of the decays to come from BSM processes. To that end, I performed a direct search of invisible Higgs decays in its most sensitive channel, vector boson fusion (VBF) production, where I identified key aspects to be improved leading the analysis to substantially increase sensitivity to this fundamental probe of DM \cite{vbfMET_CONF2020}. As analysis contact, I oversaw and guided the work of a team of graduate students and postdocs, including the day-to-day work of 5 graduate students, two of whom successfully defended their thesis. I also established a collaboration with 3 theorists to improve filtering algorithms used to generate Monte Carlo simulation of the main backgrounds and to enhance its theoretical accuracy and precision. In addition to developing a modular and scalable software package for the analysis, I recognized the importance of preserving the analysis workflow using new computing tools such as RECAST, a reproducible data analysis framework. I am using this framework to explore the Higgs sector by re-interpreting this search in terms of exotic decays of the Higgs boson to neutral long lived particles in the context of Higgs portal baryogenesis models. Over the coming months, if possible, I would like to continue leading the effort of this search to complete a legacy Run-2 paper and contribute to the combination effort of my analysis with other direct and indirect searches for invisible Higgs boson decays.

\bigskip

As there has been no definitive sign of new physics, it is important to fully exploit the LHC by looking in experimentally hard-to-reach locations that have not yet been probed in Run-2. One potential avenue that I would like to explore is the search for long-lived particles with macroscopic displacements from the primary interaction\cite{alimena2019searching}. I am incredibly excited to explore searches that capitalize on the ATLAS detector capabilities. Harvard is leading one of these searches that look for heavy charged long-lived particles based on observables related to the ionization energy loss (dE/dx) and time of flight which are sensitive to the velocity of heavy charged particles traveling significantly slower than the speed of light. I am also interested in improving standard reconstruction algorithms to ensure that tracking software and triggering are efficient at detecting displaced vertices. The higher computing cost and data size of large-radius tracking can be largely reduced by employing novel machine learning techniques to improve pattern recognition while keeping relaxed tracking impact parameter requirements. One example is to explore and extend the current outside-in tracking algorithms to use calorimeter cluster shapes more efficiently. Another is to tag long-lived particles by reconstructing displaced vertices looking at the hit patterns without doing any explicit track reconstruction. Including long-lived particle tracking into the standard reconstruction algorithms will have a considerable impact on searches for long-lived particles that decay within the inner tracking volume. Given my experience in software optimization for maximizing output within strict CPU limits in the HLT and my ambition to explore new machine learning techniques, I want to improve the long-lived particle tracking capability even further to expand the phase space probed by the LHC.



\bigskip

I plan to leverage my expertise in BSM searches, detector development, and data acquisition systems to search for long-lived particles in ATLAS and to build the ITk barrel strip detector, while assuming more responsibilities and leadership roles. I intend to take part in the Snowmass effort to develop strategies for addressing the most important questions in particle physics. I enjoy working with and guiding students to advance their physics training and education, and connecting people with different areas of expertise in order to move a team and a project forward. The work I have done during my DESY fellowship as well as in my PhD have prepared me well to take on new challenges. Harvard University plays a central role in searches for new physics and detector development, and I would be delighted to work, learn, and grow as part of the group.

\bigskip\bigskip\bigskip

\Urlmuskip=0mu plus 1mu\relax
{%\small
\fontsize{11}{14}
\selectfont
\bibliography{research_statement}}{}
\bibliographystyle{unsrt}

\end{document}
