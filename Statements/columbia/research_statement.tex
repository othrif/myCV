\documentclass[a4paper]{article}
\topmargin-2.0cm

%\usepackage{lineno}
\usepackage{amsmath}
\usepackage{amssymb}
\usepackage{amsthm}
\usepackage{amscd}
\usepackage{amsfonts}
\usepackage{graphicx}%
\usepackage{fancyhdr}
\usepackage[hyphenbreaks]{breakurl}
\usepackage[hyphens]{url}

\newcommand{\ttbarV}{\ensuremath{t\bar{t}V}}
\newcommand{\ttbarW}{\ensuremath{t\bar{t}W}}
\newcommand{\ttbarZ}{\ensuremath{t\bar{t}Z}}
\newcommand{\ifb}{\ensuremath{\mathrm{fb}^{-1}}}

\setlength{\topmargin}{-15mm}
\setlength{\textwidth}{7in}
\setlength{\oddsidemargin}{-8mm}
\setlength{\textheight}{9in}
\setlength{\footskip}{1in}


%\title{\vspace{-3cm} Research Statement}
%\author{Othmane Rifki}%\\
%University of Oklahoma\\
%E-mail: \texttt{othmane.rifki@cern.ch}}
%\date{December 2016}


\begin{document}

\thispagestyle{fancy}\lhead{\large{Research Statement}} \rhead{\large{May 22, 2020}}
\chead{{\large{\bf Othmane Rifki}}} \lfoot{} \rfoot{\bf \thepage} \cfoot{}

%\linenumbers
\fontsize{11}{14}
\selectfont
%\maketitle

My research interests lie in exploring the Standard Model (SM) of particle physics and at the same time addressing questions not answered by the SM, such as the nature of dark matter (DM). Answering these questions requires improving detector capabilities and exploring new analysis ideas to enhance the precision of our measurements and to maximize the discovery potential of the ATLAS detector. The majority of searches for beyond the SM (BSM) physics in ATLAS focuses on signatures with promptly decaying new particles, after exploring several of these BSM signatures, I would now like to also study particles with detectable displacements between their production and decay points in the long-lived particles frontier. At Columbia, I would like to develop novel techniques to search for new physics and to make important contributions to the readout electronics of the next-generation ATLAS liquid argon (LAr) calorimeter system, benefiting from and contributing to the expertise of the group in these areas.

\bigskip

So far, my research activities have spanned several areas in physics analysis and detector development. During my PhD, I performed the first searches for supersymmetry with early Run-2 data in leptonic final states after nearly doubling the center-of-mass energy and migrated the functionality of the Region of Interest Builder (RoIB), the interface that seeds the processing of every event in the high level trigger farm (HLT), from a VMEbus system to a PCIe card running in a commodity-computer. This new system has been the baseline for data taking in ATLAS since the start of 2016. As a DESY fellow, I led a search for DM in invisible Higgs boson decays, setting the strongest existing limit to date, and built a module loading station for the new all-silicon Inner Tracker (ITk) of ATLAS for the High-Luminosity LHC (HL-LHC).

\bigskip

At DESY, I have been deeply involved in the ITk strips upgrade \cite{ITKstrips}. I successfully developed a fully automated assembly system that places and glues strip modules onto a petal, the basic unit that forms the ITk end-cap disks. This system consists of mounting modules with less than 50 $\mu m$ lateral accuracy onto a petal core using a robotic gantry as a pick-and-place machine with advanced vision for identification of the sensor fiducial markers. I oversaw the purchasing of the equipment needed to build the station, totaling over $\$200$k, and exchanged with industry providers to test new solutions. I collaborated closely with other loading sites to develop and standardize the module mounting procedure and adapt a common interface to streamline production. In the process, I gained substantial knowledge of detector development such as building and testing modules, operating industrial robotic assembly, and applying computer vision for pattern finding, and in parallel improved my leadership and management skills when supervising a team of 5 including a PhD student, engineers, and technicians. In February 2020, I successfully presented the first fully loaded petal built at DESY using this automated procedure to the module loading Final Design Review, showing the petal assembly meets the desired specifications with high repeatability and increased capacity to make up for possible production delays.

\bigskip

In addition to detector development for the HL-LHC, I coordinated the search for invisible Higgs boson decays to invisible particles via the vector boson fusion (VBF) production mechanism  \cite{vbfMET_CONF2020}. This is the most sensitive search that drives the ATLAS direct probe of invisible Higgs boson decays. The rate of invisible Higgs boson decays may be significantly enhanced beyond its SM value of $\sim0.1\%$ if the Higgs boson acts as a portal to DM, a reasonable assumption given that the Higgs boson couples to massive particles and that indirect constraints allow up to 34\% of the Higgs width to come from BSM decays. I established a collaboration between theorists (Jonas Lindert, Marek Schoenherr, Stefano Augusto Pozzorini) and the ATLAS analysis team to improve filtering algorithms used in the Monte Carlo generator, which resulted in the reduction of the largest uncertainty in the previous publication by 35\%. This work also enhanced the theoretical accuracy and precision of the simulation, which allowed us to better control the main $Z\left(\nu\nu\right)$+jets background. As analysis contact, I oversaw the work of 10 graduate students, with two successful thesis defenses, and brought a number of improvements to the analysis such as optimizing the event selection and analysis object definition to increase the Higgs boson signal acceptance by 50\% and reducing the background in the signal regions by 31\%. Through these improvements, my team and I significantly increased the sensitivity to this fundamental probe of dark matter to derive an exclusion bound of the Higgs boson decay to invisible particles of 13\%. Over the coming months, if possible I would like to continue leading the effort of this search to complete a legacy Run-2 paper and contribute to the combination effort of my analysis with other direct and indirect searches for invisible Higgs boson decays.

\bigskip

Looking ahead, the next years are of great importance to prepare for Run-3 and HL-LHC in terms of new analysis ideas. As there has been no definitive sign of new physics, it is important to fully exploit the LHC by looking in experimentally hard-to-reach locations that have not yet been probed in Run-2. One potential avenue that I would like to explore is the search for long-lived particles with macroscopic displacements from the primary interaction. This type of searches offer excellent discovery prospects and arises in BSM theories proposed to address many fundamental problems in particle physics. Analogously to the SM that has particles spanning a wide range of lifetimes, long lived particles appear in many BSM theories such as gauge mediated SUSY and RPV SUSY that address the hierarchy problem, hidden valleys models with a new sector that is weakly coupled to the SM, and models that seek to incorporate dark matter \cite{alimena2019searching}. While the long-lived particles in the SM, such as b-hadrons, have well-understood experimental signatures, BSM long-lived particles offer unusual signatures that may be rejected by standard reconstruction algorithms. Dedicated methods are needed to ensure that tracking software and triggering are efficient at detecting displaced vertices. In ATLAS, tracks corresponding to these hypothetical particles are reconstructed with a special tracking configuration that only runs on a filtered part of the main physics dataset due to significantly larger computing cost and data size. This cost can be largely reduced by employing novel machine learning techniques and improved pattern recognition algorithms. One example is to explore and extend the current outside-in tracking algorithms via machine learning techniques to use calorimeter cluster shapes more efficiently. Another is to implement machine learning based algorithms to speed up the different stages of the track reconstruction such as the stage of resolving ambiguity of duplicated tracks and a  better selection of the track seeds while keeping relaxed tracking impact parameter requirements. Including long-lived particle tracking into the standard reconstruction algorithms will have a considerable impact on searches for long-lived particles that decay within the inner tracking volume. Given my experience of software optimization for maximizing output within strict CPU limits in the HLT and my ambition to explore new machine learning techniques, I want to improve the long-lived particle tracking capability even further to expand the phase space probed by the LHC.




\bigskip


Another avenue through which to improve these searches is to exploit the utility of prompt associated objects. I am interested, for instance, in continuing to explore the Higgs sector through the exotic decay of the Higgs boson to neutral long-lived particles via associated production with a vector boson. Columbia is already involved in a similar analysis looking for the Higgs boson decaying to long-lived SUSY particles that decay into a photon, which does not point back to the primary vertex, and an invisible particle. If these long-lived SUSY particles decay in the inner detector, the analysis uses the pointing and timing resolution of the LAr electromagnetic calorimeter to distinguish between prompt and displaced photons. It would be useful to investigate the use of tracking by reconstructing the converted photon vertex and its trajectory to determine its direction with respect to the primary vertex. The converted photons will have a lower statistics but an improved precision that can potentially bring additional sensitivity to displaced photon searches. I would like to explore similar non-standard ideas that capitalize on the ATLAS detector capabilities to develop new search strategies for Run-3 and beyond.

\bigskip


I plan to leverage my expertise in BSM searches, detector development, and data acquisition systems to continue searching for new physics and
prepare for the production and installation of the next generation LAr readout electronics, while assuming more responsibilities within ATLAS.  I enjoy working with and guiding students to advance their physics training and education, and connecting people with different areas of expertise in order to move a team and a project forward. The work I have done during my DESY fellowship as well as in my PhD have prepared me well to take on new challenges. Columbia plays a central role in both searches for new physics and hardware upgrades, and I would be delighted to work, learn, and grow as part of the group.



\Urlmuskip=0mu plus 1mu\relax
{%\small
\fontsize{11}{14}
\selectfont
\bibliography{research_statement}}{}
\bibliographystyle{unsrt}

\end{document}
